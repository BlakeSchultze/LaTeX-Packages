%-------------------------------------------------------------------------------------------------------------------------------%
%~~~~~~~~~~~~~~~~~~~~~~~~~~~~~~~~~~~~~~~~~~~~~~~~~~~~~~~~~~~~~~~~~~~~~~~~~~~~~~~~~~~~~~~~~~~~~~~~~~~~~~~~~~~~~~~~~~~~~~~~~~~~~~~%
%~~~~~~~~~~~~~~~~~~~~~~~~~~~~~~~~~~~~~~~~~~~~~~~~~~~~~~~~~~~~~~~~~~~~~~~~~~~~~~~~~~~~~~~~~~~~~~~~~~~~~~~~~~~~~~~~~~~~~~~~~~~~~~~%
%~~~~~~~~~~~~~~~~~~~~~~~~~~~~~~~~~~~~~~~~~~~~~~~~~~~~~~~~~~~~~~~~~~~~~~~~~~~~~~~~~~~~~~~~~~~~~~~~~~~~~~~~~~~~~~~~~~~~~~~~~~~~~~~%
%-------------------------------------------------------------------------------------------------------------------------------%
%\begin{tcbenclosure}[type=eqnlist]%
\begin{eqnlistbox}
%-------------------------------------------------------------------------------------------------------------------------------%
%-------------------------------------------------------------------------------------------------------------------------------%
%-------------------------------------------------------------------------------------------------------------------------------%
\begin{tcbeqnlist}[style=blue, beforeskip=true, skipsize=big, goldtrim=false]
    \tcbeqnlistboxentry
        {\text{Mean} = \mu_x = 1 > 0}
        {\implies\text{$f_x(x)$ is shifted to the right of the origin}}
        \\
        \tcbeqnlistboxentry
        {\text{Variance} = \var{x(\zeta)} \triangleq\sigma_x^2 = 1 > 0}
        {\implies\text{$f_x(x)$ has an equal spread in values as a standard Gaussian}}
        \\
        \tcbeqnlistboxentry
        {\text{Skewness} \triangleq\tilde{\kappa}_x^{(3)} = 2 > 0}
        {\implies\text{$f_x(x)$ leans right.}}
        \\
        \tcbeqnlistboxentry
        {\text{Kurtosis} = \tilde{\kappa}_x^{(4)} = 6 > 0}
        {\implies\text{$f_x(x)$ has a much sharper peak than a standard Gaussian}}
\end{tcbeqnlist}
%-------------------------------------------------------------------------------------------------------------------------------%
\begin{tcbeqnlist}[style=solution, beforeskip=true, goldtrim=false]
       \tcbeqnlistboxentry
           {\mean{x} }
           {= \dfrac{1}{2}\quad \textbf{: constant}}
            \\
            \tcbeqnlistboxentry
               {r_x(n_1,n_2) }
               {=\dfrac{1}{3}\quad \textbf{: constant}}
            \\
            \tcbeqnlistboxentry
               {\Longrightarrow\quad X(t) }
               {= A(\zeta) \textbf{ \ul{is} WSS}}
\end{tcbeqnlist}
%-------------------------------------------------------------------------------------------------------------------------------%
\begin{tcbeqnlist}[style=solution,colorscheme=tcbDarkBlueScheme, beforeskip=true, title=Solution]
    \tcbeqnlistboxentry
        {\mean{x} }
        {= \dfrac{1}{2}\quad \textbf{: constant}}
    \\
    \tcbeqnlistboxentry
        {r_x(n_1,n_2) }
        {=\dfrac{1}{3}\quad \textbf{: constant}}
    \\
    \tcbeqnlistboxentry
        {\Longrightarrow\quad X(t) }
        {= A(\zeta) \textbf{ \ul{is} WSS}}
\end{tcbeqnlist}
%-------------------------------------------------------------------------------------------------------------------------------%
%-------------------------------------------------------------------------------------------------------------------------------%
%-------------------------------------------------------------------------------------------------------------------------------%
\end{eqnlistbox}
%%\end{tcbenclosure}
%-------------------------------------------------------------------------------------------------------------------------------%
%~~~~~~~~~~~~~~~~~~~~~~~~~~~~~~~~~~~~~~~~~~~~~~~~~~~~~~~~~~~~~~~~~~~~~~~~~~~~~~~~~~~~~~~~~~~~~~~~~~~~~~~~~~~~~~~~~~~~~~~~~~~~~~~%
%~~~~~~~~~~~~~~~~~~~~~~~~~~~~~~~~~~~~~~~~~~~~~~~~~~~~~~~~~~~~~~~~~~~~~~~~~~~~~~~~~~~~~~~~~~~~~~~~~~~~~~~~~~~~~~~~~~~~~~~~~~~~~~~%
%~~~~~~~~~~~~~~~~~~~~~~~~~~~~~~~~~~~~~~~~~~~~~~~~~~~~~~~~~~~~~~~~~~~~~~~~~~~~~~~~~~~~~~~~~~~~~~~~~~~~~~~~~~~~~~~~~~~~~~~~~~~~~~~%
%-------------------------------------------------------------------------------------------------------------------------------%
\begin{problem}[3](8)%
    For each of the following, determine whether the random process is (1) WSS or (2) m.s. ergodic in the mean.
\end{problem}
%-------------------------------------------------------------------------------------------------------------------------------%
\begin{problistbox}
%\begin{tcbenclosure}[type=problem]%,colorscheme=tcbDarkBlueScheme]
%-------------------------------------------------------------------------------------------------------------------------------%
%-------------------------------------------------------------------------------------------------------------------------------%
%-------------------------------------------------------------------------------------------------------------------------------%
\begin{tcbeqnlist}[style=blue, beforeskip=true, skipsize=big, goldtrim=false]
    \tcbeqnlistboxentry
        {\text{Mean} = \mu_x = 1 > 0}
        {\implies\text{$f_x(x)$ is shifted to the right of the origin}}
        \\
        \tcbeqnlistboxentry
        {\text{Variance} = \var{x(\zeta)} \triangleq\sigma_x^2 = 1 > 0}
        {\implies\text{$f_x(x)$ has an equal spread in values as a standard Gaussian}}
        \\
        \tcbeqnlistboxentry
        {\text{Skewness} \triangleq\tilde{\kappa}_x^{(3)} = 2 > 0}
        {\implies\text{$f_x(x)$ leans right.}}
        \\
        \tcbeqnlistboxentry
        {\text{Kurtosis} = \tilde{\kappa}_x^{(4)} = 6 > 0}
        {\implies\text{$f_x(x)$ has a much sharper peak than a standard Gaussian}}
\end{tcbeqnlist}
%-------------------------------------------------------------------------------------------------------------------------------%
\begin{tcbeqnlist}[style=solution, beforeskip=true, goldtrim=false]
       \tcbeqnlistboxentry
           {\mean{x} }
           {= \dfrac{1}{2}\quad \textbf{: constant}}
            \\
            \tcbeqnlistboxentry
               {r_x(n_1,n_2) }
               {=\dfrac{1}{3}\quad \textbf{: constant}}
            \\
            \tcbeqnlistboxentry
               {\Longrightarrow\quad X(t) }
               {= A(\zeta) \textbf{ \ul{is} WSS}}
\end{tcbeqnlist}
%-------------------------------------------------------------------------------------------------------------------------------%
\begin{tcbeqnlist}[style=solution,colorscheme=tcbDarkBlueScheme, beforeskip=true, title=Solution]
    \tcbeqnlistboxentry
        {\mean{x} }
        {= \dfrac{1}{2}\quad \textbf{: constant}}
    \\
    \tcbeqnlistboxentry
        {r_x(n_1,n_2) }
        {=\dfrac{1}{3}\quad \textbf{: constant}}
    \\
    \tcbeqnlistboxentry
        {\Longrightarrow\quad X(t) }
        {= A(\zeta) \textbf{ \ul{is} WSS}}
\end{tcbeqnlist}
%-------------------------------------------------------------------------------------------------------------------------------%
%-------------------------------------------------------------------------------------------------------------------------------%
%-------------------------------------------------------------------------------------------------------------------------------%
\end{problistbox}
%\end{tcbenclosure}
%-------------------------------------------------------------------------------------------------------------------------------%
%~~~~~~~~~~~~~~~~~~~~~~~~~~~~~~~~~~~~~~~~~~~~~~~~~~~~~~~~~~~~~~~~~~~~~~~~~~~~~~~~~~~~~~~~~~~~~~~~~~~~~~~~~~~~~~~~~~~~~~~~~~~~~~~%
%~~~~~~~~~~~~~~~~~~~~~~~~~~~~~~~~~~~~~~~~~~~~~~~~~~~~~~~~~~~~~~~~~~~~~~~~~~~~~~~~~~~~~~~~~~~~~~~~~~~~~~~~~~~~~~~~~~~~~~~~~~~~~~~%
%~~~~~~~~~~~~~~~~~~~~~~~~~~~~~~~~~~~~~~~~~~~~~~~~~~~~~~~~~~~~~~~~~~~~~~~~~~~~~~~~~~~~~~~~~~~~~~~~~~~~~~~~~~~~~~~~~~~~~~~~~~~~~~~%
%-------------------------------------------------------------------------------------------------------------------------------%
%\begin{tcbenclosure}[type=solution]%,colorscheme=tcbDarkBlueScheme]
\begin{sollistbox}%[colorscheme=tcbDarkBlueScheme]%,
%-------------------------------------------------------------------------------------------------------------------------------%
%-------------------------------------------------------------------------------------------------------------------------------%
%-------------------------------------------------------------------------------------------------------------------------------%
\begin{tcbeqnlist}[style=blue, beforeskip=true, skipsize=big, goldtrim=false]
    \tcbeqnlistboxentry
        {\text{Mean} = \mu_x = 1 > 0}
        {\implies\text{$f_x(x)$ is shifted to the right of the origin}}
        \\
        \tcbeqnlistboxentry
        {\text{Variance} = \var{x(\zeta)} \triangleq\sigma_x^2 = 1 > 0}
        {\implies\text{$f_x(x)$ has an equal spread in values as a standard Gaussian}}
        \\
        \tcbeqnlistboxentry
        {\text{Skewness} \triangleq\tilde{\kappa}_x^{(3)} = 2 > 0}
        {\implies\text{$f_x(x)$ leans right.}}
        \\
        \tcbeqnlistboxentry
        {\text{Kurtosis} = \tilde{\kappa}_x^{(4)} = 6 > 0}
        {\implies\text{$f_x(x)$ has a much sharper peak than a standard Gaussian}}
\end{tcbeqnlist}
%-------------------------------------------------------------------------------------------------------------------------------%
\begin{tcbeqnlist}[style=solution, beforeskip=true, goldtrim=false]
       \tcbeqnlistboxentry
           {\mean{x} }
           {= \dfrac{1}{2}\quad \textbf{: constant}}
            \\
            \tcbeqnlistboxentry
               {r_x(n_1,n_2) }
               {=\dfrac{1}{3}\quad \textbf{: constant}}
            \\
            \tcbeqnlistboxentry
               {\Longrightarrow\quad X(t) }
               {= A(\zeta) \textbf{ \ul{is} WSS}}
\end{tcbeqnlist}
%-------------------------------------------------------------------------------------------------------------------------------%
\begin{tcbeqnlist}[style=solution,colorscheme=tcbDarkBlueScheme, beforeskip=true, title=Solution]
    \tcbeqnlistboxentry
        {\mean{x} }
        {= \dfrac{1}{2}\quad \textbf{: constant}}
    \\
    \tcbeqnlistboxentry
        {r_x(n_1,n_2) }
        {=\dfrac{1}{3}\quad \textbf{: constant}}
    \\
    \tcbeqnlistboxentry
        {\Longrightarrow\quad X(t) }
        {= A(\zeta) \textbf{ \ul{is} WSS}}
\end{tcbeqnlist}
%-------------------------------------------------------------------------------------------------------------------------------%
%-------------------------------------------------------------------------------------------------------------------------------%
%-------------------------------------------------------------------------------------------------------------------------------%
\end{sollistbox}
%\end{tcbenclosure}
%-------------------------------------------------------------------------------------------------------------------------------%
%~~~~~~~~~~~~~~~~~~~~~~~~~~~~~~~~~~~~~~~~~~~~~~~~~~~~~~~~~~~~~~~~~~~~~~~~~~~~~~~~~~~~~~~~~~~~~~~~~~~~~~~~~~~~~~~~~~~~~~~~~~~~~~~%
%~~~~~~~~~~~~~~~~~~~~~~~~~~~~~~~~~~~~~~~~~~~~~~~~~~~~~~~~~~~~~~~~~~~~~~~~~~~~~~~~~~~~~~~~~~~~~~~~~~~~~~~~~~~~~~~~~~~~~~~~~~~~~~~%
%~~~~~~~~~~~~~~~~~~~~~~~~~~~~~~~~~~~~~~~~~~~~~~~~~~~~~~~~~~~~~~~~~~~~~~~~~~~~~~~~~~~~~~~~~~~~~~~~~~~~~~~~~~~~~~~~~~~~~~~~~~~~~~~%
%-------------------------------------------------------------------------------------------------------------------------------%
%\begin{tcbenclosure}[type=list,colorscheme=tcbDarkBlueScheme]
\begin{listbox}[colorscheme=tcbDarkBlueScheme]
%-------------------------------------------------------------------------------------------------------------------------------%
%-------------------------------------------------------------------------------------------------------------------------------%
%-------------------------------------------------------------------------------------------------------------------------------%
\centering
%-------------------------------------------------------------------------------------------------------------------------------%
\begin{tcblist}[style=darkblue, overwritegradient=true, goldtrim=true, width=14.5cm]
   \tcblistboxentry
       {Its mean is a constant independent of $n$, that is,}
       {\E{x(n)} =\mean{x}}
   \tcblistboxentry
       {Its variance is also a constant independent of $n$, that is}
       {\var{x(n)} =  \variance{x}}
   \tcblistboxentry
       {Its autocorrelation depends only on the distance $\ell = n_1-n_2$, called the lag, that is}
       {r_{x}(n_1, n_2) = r_x(n_1-n_2)=r_x(\ell) = \E{x(n)x^*(n+l)} = \E{x(n+\ell)x^*(n)}}
\end{tcblist}
%-------------------------------------------------------------------------------------------------------------------------------%
\begin{tcblist}[style=darkblue, title={A random process $x(n)$ is \emph{wide-sense stationary} (WSS) if:}]
   \tcblistboxentry
       {Its mean is a constant independent of $n$, that is,}
       {\E{x(n)} =\mean{x}}
   \tcblistboxentry
       {Its variance is also a constant independent of $n$, that is}
       {\var{x(n)} =  \variance{x}}
   \tcblistboxentry
       {Its autocorrelation depends only on the distance $\ell = n_1-n_2$, called the lag, that is}
       {r_{x}(n_1, n_2) = r_x(n_1-n_2)=r_x(\ell) = \E{x(n)x^*(n+l)} = \E{x(n+\ell)x^*(n)}}
\end{tcblist}
%-------------------------------------------------------------------------------------------------------------------------------%
\begin{tcblist}[style=maroon, width=14.5cm, title={A random process $x(n)$ is \emph{wide-sense stationary} (WSS) if:}]
   \tcblistboxentry
       {Its mean is a constant independent of $n$, that is,}
       {\E{x(n)} =\mean{x}}
   \tcblistboxentry
       {Its variance is also a constant independent of $n$, that is}
       {\var{x(n)} =  \variance{x}}
   \tcblistboxentry
       {Its autocorrelation depends only on the distance $\ell = n_1-n_2$, called the lag, that is}
       {r_{x}(n_1, n_2) = r_x(n_1-n_2)=r_x(\ell) = \E{x(n)x^*(n+l)} = \E{x(n+\ell)x^*(n)}}
\end{tcblist}
%-------------------------------------------------------------------------------------------------------------------------------%
\begin{tcblist}[style=maroon, goldtrim=true, title={Mean Sense (M.S.) Ergodic in the Mean:}]
  \tcblistboxentry{A random process |x(n)| is ergodic in the mean, i.e. M.S. ergodic, if}
  {
      \timeavg{x(n)}
      = \E{x(n)}=\mean{x}
      = \dfrac{1}{2N+1}
      \displaystyle\sum\limits_{-N}^{N} x(n)
  }
  \tcblistboxentry{A random process |x(n)| is ergodic in the mean, i.e. M.S. ergodic, if}
  {
      \timeavg{x(n)}
      = \E{x(n)}=\mean{x}
      = \dfrac{1}{2N+1}
      \displaystyle\sum\limits_{-N}^{N} x(n)
  }
  \tcblistboxentry{A random process |x(n)| is ergodic in the mean, i.e. M.S. ergodic, if}
  {
      \timeavg{x(n)}
      = \E{x(n)}=\mean{x}
      = \dfrac{1}{2N+1}
      \sum\limits_{-N}^{N} x(n)
  }
\end{tcblist}
%-------------------------------------------------------------------------------------------------------------------------------%
\begin{tcblist}[style=maroon, goldtrim=true, title={A random process $x(n)$ is \emph{wide-sense stationary} (WSS) if:}]
   \tcblistboxentry
       {Its mean is a constant independent of $n$, that is,}
       {\E{x(n)} =\mean{x}}
   \tcblistboxentry
       {Its variance is also a constant independent of $n$, that is}
       {\var{x(n)} =  \variance{x}}
   \tcblistboxentry
       {Its autocorrelation depends only on the distance $\ell = n_1-n_2$, called the lag, that is}
       {r_{x}(n_1, n_2) = r_x(n_1-n_2)=r_x(\ell) = \E{x(n)x^*(n+l)} = \E{x(n+\ell)x^*(n)}}
\end{tcblist}
%-------------------------------------------------------------------------------------------------------------------------------%
\begin{tcblist}[ title={Mean Sense (M.S.) Ergodic in the Mean:}]
  \tcblistboxentry{A random process |x(n)| is ergodic in the mean, i.e. M.S. ergodic, if}
  {
      \timeavg{x(n)}
      = \E{x(n)}=\mean{x}
      = \dfrac{1}{2N+1}
      \displaystyle\sum\limits_{-N}^{N} x(n)
  }
  \tcblistboxentry{A random process |x(n)| is ergodic in the mean, i.e. M.S. ergodic, if}
  {
      \timeavg{x(n)}
      = \E{x(n)}=\mean{x}
      = \dfrac{1}{2N+1}
      \displaystyle\sum\limits_{-N}^{N} x(n)
  }
  \tcblistboxentry{A random process |x(n)| is ergodic in the mean, i.e. M.S. ergodic, if}
  {
      \timeavg{x(n)}
      = \E{x(n)}=\mean{x}
      = \dfrac{1}{2N+1}
      \sum\limits_{-N}^{N} x(n)
  }
\end{tcblist}
%-------------------------------------------------------------------------------------------------------------------------------%
\begin{tcblist}[style=darkred, width=15.5cm, beforeskip=true, width=14.5cm, title={Mean Sense (M.S.) Ergodic in the Mean:}]
  \tcblistboxentry{A random process |x(n)| is ergodic in the mean, i.e. M.S. ergodic, if}
  {
      \timeavg{x(n)}
      = \E{x(n)}=\mean{x}
      = \dfrac{1}{2N+1}
      \displaystyle\sum\limits_{-N}^{N} x(n)
  }
  \tcblistboxentry{A random process |x(n)| is ergodic in the mean, i.e. M.S. ergodic, if}
  {
      \timeavg{x(n)}
      = \E{x(n)}=\mean{x}
      = \dfrac{1}{2N+1}
      \displaystyle\sum\limits_{-N}^{N} x(n)
  }
  \tcblistboxentry{A random process |x(n)| is ergodic in the mean, i.e. M.S. ergodic, if}
  {
      \timeavg{x(n)}
      = \E{x(n)}=\mean{x}
      = \dfrac{1}{2N+1}
      \sum\limits_{-N}^{N} x(n)
  }
\end{tcblist}
%-------------------------------------------------------------------------------------------------------------------------------%
\begin{tcblist}[style=darkgreen, title={Mean Sense (M.S.) Ergodic in the Mean:}]
  \tcblistboxentry{A random process |x(n)| is ergodic in the mean, i.e. M.S. ergodic, if}
  {
      \timeavg{x(n)}
      = \E{x(n)}=\mean{x}
      = \dfrac{1}{2N+1}
      \displaystyle\sum\limits_{-N}^{N} x(n)
  }
  \tcblistboxentry{A random process |x(n)| is ergodic in the mean, i.e. M.S. ergodic, if}
  {
      \timeavg{x(n)}
      = \E{x(n)}=\mean{x}
      = \dfrac{1}{2N+1}
      \displaystyle\sum\limits_{-N}^{N} x(n)
  }
  \tcblistboxentry{A random process |x(n)| is ergodic in the mean, i.e. M.S. ergodic, if}
  {
      \timeavg{x(n)}
      = \E{x(n)}=\mean{x}
      = \dfrac{1}{2N+1}
      \sum\limits_{-N}^{N} x(n)
  }
\end{tcblist}
%-------------------------------------------------------------------------------------------------------------------------------%
\begin{tcblist}[style=darkgreen, goldtrim=true, title={Mean Sense (M.S.) Ergodic in the Mean:}]
  \tcblistboxentry{A random process |x(n)| is ergodic in the mean, i.e. M.S. ergodic, if}
  {
      \timeavg{x(n)}
      = \E{x(n)}=\mean{x}
      = \dfrac{1}{2N+1}
      \displaystyle\sum\limits_{-N}^{N} x(n)
  }
  \tcblistboxentry{A random process |x(n)| is ergodic in the mean, i.e. M.S. ergodic, if}
  {
      \timeavg{x(n)}
      = \E{x(n)}=\mean{x}
      = \dfrac{1}{2N+1}
      \displaystyle\sum\limits_{-N}^{N} x(n)
  }
  \tcblistboxentry{A random process |x(n)| is ergodic in the mean, i.e. M.S. ergodic, if}
  {
      \timeavg{x(n)}
      = \E{x(n)}=\mean{x}
      = \dfrac{1}{2N+1}
      \sum\limits_{-N}^{N} x(n)
  }
\end{tcblist}
%-------------------------------------------------------------------------------------------------------------------------------%
\begin{tcblist}[style=darkgreen, goldtrim=true, title={}]]
  \tcblistboxentry{A random process |x(n)| is ergodic in the mean, i.e. M.S. ergodic, if}
  {
      \timeavg{x(n)}
      = \E{x(n)}=\mean{x}
      = \dfrac{1}{2N+1}
      \displaystyle\sum\limits_{-N}^{N} x(n)
  }
  \tcblistboxentry{A random process |x(n)| is ergodic in the mean, i.e. M.S. ergodic, if}
  {
      \timeavg{x(n)}
      = \E{x(n)}=\mean{x}
      = \dfrac{1}{2N+1}
      \displaystyle\sum\limits_{-N}^{N} x(n)
  }
  \tcblistboxentry{A random process |x(n)| is ergodic in the mean, i.e. M.S. ergodic, if}
  {
      \timeavg{x(n)}
      = \E{x(n)}=\mean{x}
      = \dfrac{1}{2N+1}
      \sum\limits_{-N}^{N} x(n)
  }
\end{tcblist}
%-------------------------------------------------------------------------------------------------------------------------------%
%\end{tcbenclosure}
\end{listbox}
\endinput 