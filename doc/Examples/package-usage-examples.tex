%-------------------------------------------------------------------------------------------------------------------------------%
\Part{Package Usage Examples}
%-------------------------------------------------------------------------------------------------------------------------------%
\Topic{Example Application: Data Storage Documentation}
%-------------------------------------------------------------------------------------------------------------------------------%
%\clearpage%
\Chapter*{Directory Usage}
%\gls{computers}
\begin{tcbsection}[type=legend, underlined=false]
    \item \legendentry[documentation-constdir]{Green}{directories whose names do not change}%
    \item \legendentry[documentation-dir]{Brown}{directories w/ parameter dependent naming (e.g. object name, date, etc.)}%
    \item \legendentry[documentation-file]{\emph{Italic $+$ Royal Blue}}{data/image file}%
    \item \legendentry[documentation-fileset]{\emph{Italic + Dark Blue}}{set of data/image files}%
\end{tcbsection}
%Now this is in the middle of a paragraph so it better be oriented correctly dammit\index{a} and then \index{ab} and then\index{ab-c} and then \tcbinlinebashbox{cd Documents}
\input{C:/Users/Blake/Documents/LaTeX/Documentation/PackageUsage/Kodiak_directory_usage.tex}
\idxTestContent
\glsTestContent
\abbrevTestContent
\input{C:/Users/Blake/Documents/LaTeX/Documentation/PackageUsage/Tardis_directory_usage.tex}
%*******************************************************************************************************************************%
%*******************************************************************************************************************************%
%*******************************************************************************************************************************%
\Chapter{Organizational Scheme}
%\Section{Organization}%[Description of directory hierarchies relevant to storage of code (see \autohyperlink{]
Now this is in the middle of a paragrap
\glsTestContent
\idxTestContent
\abbrevTestContent
%-------------------------------------------------------------------------------------------------------------------------------%
%-------------------------------------------------------------------------------------------------------------------------------%
%-------------------------------------------------------------------------------------------------------------------------------%
%\Section!{Code Organization}%[Description of directory hierarchies relevant to storage of code (see \autohyperlink{]
\Section{Code Organization}%[Description of directory hierarchies relevant to storage of code (see \autohyperlink{]
%-------------------------------------------------------------------------------------------------------------------------------%
\input{C:/Users/Blake/Documents/LaTeX/Documentation/PackageUsage/pct_code_hierarchy.tex}
%-------------------------------------------------------------------------------------------------------------------------------%
%-------------------------------------------------------------------------------------------------------------------------------%
%-------------------------------------------------------------------------------------------------------------------------------%
%\clearpage%
\Section{Data Organization}
%\Section*!{Data Organization}
%-------------------------------------------------------------------------------------------------------------------------------%
\input{C:/Users/Blake/Documents/LaTeX/Documentation/PackageUsage/organized_data_hierarchy.tex}
\input{C:/Users/Blake/Documents/LaTeX/Documentation/PackageUsage/raw_data_hierarchy.tex}
\input{C:/Users/Blake/Documents/LaTeX/Documentation/PackageUsage/preprocessed_data_hierarchy.tex}
\input{C:/Users/Blake/Documents/LaTeX/Documentation/PackageUsage/user_data_hierarchy.tex}
\input{C:/Users/Blake/Documents/LaTeX/Documentation/PackageUsage/reconstruction_data_hierarchy.tex}
\gls{naiive}\\
\gls{computer}\\
%-------------------------------------------------------------------------------------------------------------------------------%
%-------------------------------------------------------------------------------------------------------------------------------%
%-------------------------------------------------------------------------------------------------------------------------------%
\Chapter{Code Hierarchy Diagrams}
First line after chapter
\index{hello}
\idxTestContent
\input{C:/Users/Blake/Documents/LaTeX/Documentation/PackageUsage/pCT_Code_Hierarchy_Diagram.tex}
\input{C:/Users/Blake/Documents/LaTeX/Documentation/PackageUsage/pCT_Data_Hierarchy_Diagram.tex}
%-------------------------------------------------------------------------------------------------------------------------------%
%-------------------------------------------------------------------------------------------------------------------------------%
%-------------------------------------------------------------------------------------------------------------------------------%
\Chapter{GitHub Sources}
\input{C:/Users/Blake/Documents/LaTeX/Documentation/PackageUsage/GitHub_hierarchy.tex}
\abbrevTestContent
%-------------------------------------------------------------------------------------------------------------------------------%
%-------------------------------------------------------------------------------------------------------------------------------%
%-------------------------------------------------------------------------------------------------------------------------------%
%%\clearpage
\Chapter{File Lists}
\input{C:/Users/Blake/Documents/LaTeX/Documentation/PackageUsage/recon_data_files.tex}
\input{C:/Users/Blake/Documents/LaTeX/Documentation/PackageUsage/file_list.tex}%\clearpage\setcounter{page}{0}\thispagestyle{fancy}\pagenumbering{roman}
%-------------------------------------------------------------------------------------------------------------------------------%
%-------------------------------------------------------------------------------------------------------------------------------%
%-------------------------------------------------------------------------------------------------------------------------------%
\endinput