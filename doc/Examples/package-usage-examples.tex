%-------------------------------------------------------------------------------------------------------------------------------%
\Part{Package Usage Examples}
%-------------------------------------------------------------------------------------------------------------------------------%
\Topic{Example Application: Data Storage Documentation}
%-------------------------------------------------------------------------------------------------------------------------------%
%\clearpage%
\Chapter*{Directory Usage}
%\gls{computers}
\begin{tcbsection}[type=legend, underlined=false]
    \item \legendentry[documentation-constdir]{Green}{directories whose names do not change}%
    \item \legendentry[documentation-dir]{Brown}{directories w/ parameter dependent naming (e.g. object name, date, etc.)}%
    \item \legendentry[documentation-file]{\emph{Italic $+$ Royal Blue}}{data/image file}%
    \item \legendentry[documentation-fileset]{\emph{Italic + Dark Blue}}{set of data/image files}%
\end{tcbsection}
%Now this is in the middle of a paragraph so it better be oriented correctly dammit\index{a} and then \index{ab} and then\index{ab-c} and then \tcbinlinebashbox{cd Documents}
%-----------------------------------------------------------------------------------------------------------------------------------------------------------------------%
\Section{Kodiak Directories}[Description and purpose/usage of the private user and shared directories on Kodiak]
%-----------------------------------------------------------------------------------------------------------------------------------------------------------------------%
\begin{tcbenvironment}|Kodiak|
%-----------------------------------------------------------------------------------------------------------------------------------------------------------------------%
%------------------------------------------------------- ion/... -------------------------------------------------------------%
%-----------------------------------------------------------------------------------------------------------------------------------------------------------------------%
\begin{tcbparbox}-|\dirsep ion\dirsep$\dots$|%
This parent directory is dedicated to the storage of all files associated with proton and ion therapy research and is on the mounted network-attached storage (NAS) device.  Users have write permissions to their private subdirectories and to share data/code, users move it to their private ``staging'' directory (with the appropriate file naming/organization) and it is then validated before moving it to the corresponding shared public subdirectory.  This is intended to prevent inappropriate additions and accidental modifications/deletions of existing code/data.  The organization of submitted code/data should include all directories below \docentry[constdir]{ion} so the entire hierarchy can be merged with the existing subdirectories automatically, easing manual merging for administrators (who will not know where to move the contents) and simplifying automated data verification script development.  

The data in this directory is located on a network storage device and can be accessed from all Kodiak and Tardis cluster nodes.  The device is also backed up to    tape drive periodically to prevent permanent loss of data in the event of drive failure.
\end{tcbparbox}
%-----------------------------------------------------------------------------------------------------------------------------------------------------------------------%
%---------------------------------------------------- ion/home/<username>/... -----------------------------------------------------%
%-----------------------------------------------------------------------------------------------------------------------------------------------------------------------%
\begin{tcbparbox}|\dirsep ion\dirsep home\dirsep\caratenclosed*{username}\dirsep$\dots$|
\bfdash This is a user's private home directory where the files associated with their account are stored (e.g., \docentry[file]{\texttt{.bash\_profile}}, \docentry[file]{\texttt{.bash\_history}}, etc.) and is the default login directory.  Each user only has access to their personal directory, but because it is on the network storage device, it can be accessed from each of the Kodiak/Tardis nodes. Now that the home directories have been moved to \docentry[constdir]{ion}, they no longer have a limited storage capacity, so users may run code and write the resulting output data/images to this directory.  Note that as a subdirectory of \docentry[constdir]{ion}, the data in this directory will automatically be backed up to tape drive so it is recoverable in case of data corruption or drive failure.
\end{tcbparbox}
%-----------------------------------------------------------------------------------------------------------------------------------------------------------------------%
%-------------------------------------------------- /data/<username>/... -------------------------------------------------%
%-----------------------------------------------------------------------------------------------------------------------------------------------------------------------%
\begin{tcbparbox}|\dirsep data\dirsep\caratenclosed*{username}\dirsep$\dots$|
\bfdash These private data directories can be used as an alternative to \docentry[dir]{ion\dirsep home\dirsep\caratenclosed*{username}} for storing input data for code/program execution and as the destination for the resulting output data generated.  As subdirectories of \docentry[constdir]{data}, the contents of these directories are backed up periodically, so these can also be used for long term data storage.
\end{tcbparbox}
%-----------------------------------------------------------------------------------------------------------------------------------------------------------------------%
%---------------------------------------------------- ion/incoming/<username> ---------------------------------------------------%
%-----------------------------------------------------------------------------------------------------------------------------------------------------------------------%
\begin{tcbparbox}|\dirsep ion\dirsep incoming\dirsep\caratenclosed*{username}\dirsep$\dots$|
\bfdash These private directories are used to upload data to the Baylor server prior to moving it to the intended destination.  When the uploaded data is intended to be shared with the collaboration, the directory should be used to rename and organize the data files according to the naming/organizational scheme before moving it to a user's private \docentry[constdir]{staging} directory, from which an administrator will validate and move the data to the appropriate shared directory.
\end{tcbparbox}
%-----------------------------------------------------------------------------------------------------------------------------------------------------------------------%
%------------------------------------------------- ion/staging/<username> -----------------------------------------------------%
%-----------------------------------------------------------------------------------------------------------------------------------------------------------------------%
\begin{tcbparbox}|\dirsep ion\dirsep staging\dirsep\caratenclosed*{username}\dirsep$\dots$|
\bfdash These directories are used to submit code/data for sharing with the collaboration.  Since administrators are typically unfamiliar with the naming/organizational scheme for shared data, users must first rename/organize the data as needed to create the entire hierarchy of directories corresponding to the full destination path, including all subdirectories below \docentry[constdir]{ion}.  Administrators need not know the destination path or understand the organization but can then simply move the entire hierarchy and the contents of existing directories such as \docentry[constdir]{ion} will automatically be merged and the new data/directories added.  To simplify the creation of these hierarchies and ensure consistency by removing manual naming/organization, bash scripts/functions have been developed to organize data and move it to a user's \docentry[constdir]{staging} directory by passing the requisite information as execution parameters (e.g., phantom, run date/\#/tag(s), etc.).
\end{tcbparbox}
%-----------------------------------------------------------------------------------------------------------------------------------------------------------------------%
%------------------------------------------------------------ ion/pCT_data/... -------------------------------------------------%
%-----------------------------------------------------------------------------------------------------------------------------------------------------------------------%
\begin{tcbparbox}|\dirsep ion\dirsep pCT\_data\dirsep$\dots$|
\bfdash This directory is where the raw, preprocessed, projection, and reconstruction data/images are moved to make them available to the other pCT users.  Each type of data is stored in separate subdirectories and soft links to this data are created and organized in a directory hierarchy indicating their input/output data dependencies.  The directory/file naming and organizational scheme for each type of data and the soft links are outlined in the next section of this document.  Data/images should only be moved to this shared directory after having been verified as valid/accurate and having been organized appropriately.
\end{tcbparbox}
%-----------------------------------------------------------------------------------------------------------------------------------------------------------------------%
%-------------------------------------------------- ion/pCT_data/pCT_Documentation ---------------------------------------------%
%-----------------------------------------------------------------------------------------------------------------------------------------------------------------------%
\begin{tcbparbox}|\dirsep ion\dirsep pCT\_data\dirsep pCT\_Documentation\dirsep $\dots$|
\bfdash Documentation relevant to pCT is stored in this directory, such as descriptions of the data format, coordinate system, and phantoms and pCT related publications (including student theses/dissertations).  This is a GitHub managed local repository allowing everyone to \docentry![subcategory]{push} contributions to the repository and \docentry![subcategory]{pull} updates/additions from others into their own local clone ensuring everyone has access to the latest information.
\end{tcbparbox}
%-----------------------------------------------------------------------------------------------------------------------------------------------------------------------%
%------------------------------------------------------- ion/pCT_code/... -------------------------------------------------------%
%-----------------------------------------------------------------------------------------------------------------------------------------------------------------------%
\begin{tcbparbox}|\dirsep ion\dirsep pCT\_code\dirsep $\dots$|
\bfdash This directory is used to store permanent and semi-permanent pCT source code, from data acquisition to image reconstruction and analysis of reconstructed images.  It contains clones of GitHub repositories as well as user's personal versions of programs they want to make available to other users (otherwise users should keep their code in their private directories) organized by program type (Preprocessing/Reconstruction/etc.) with subdirectories for each user.
\end{tcbparbox}
%-----------------------------------------------------------------------------------------------------------------------------------------------------------------------%
%------------------------------------------ ion/pCT_code/git/<GitHub account>/<GitHub repo>/... -----------------------------------%
%-----------------------------------------------------------------------------------------------------------------------------------------------------------------------%
\begin{tcbparbox}|\csvdirpath*{ion,{pCT\_code},git,\vardir*{GitHub account},\vardir*{GitHub repository}}+|
%|\dirsep ion\dirsep pCT\_code\dirsep git\dirsep\caratenclosed*{GitHub account}\dirsep\caratenclosed*{GitHub repository}\dirsep $\dots$|
\bfdash This directory contains clones of the available pCT GitHub accounts and repositories, with parent directories for each GitHub account and subdirectories for each of their repositories.  Each program repository has a \docentry[gitbranch]{master} branch, which typically corresponds to the current release version (though there may also be a branch like \docentry[gitbranch]{release} used instead) and each of the program's developers will typically have their own branch which they can use to develop and test new ideas/features.  The group of developers of a program should decide amongst themselves what the process will be for approving merges with the \docentry[gitbranch]{master/release} branch and when to release a new version of the program, which may include the results of several separate merges.  Users accessing the \docentry[gitbranch]{master/release} branch of these clones should execute \tcbinlinebashbox!{git pull --rebase} prior to using the code to ensure it is updated to its latest version.\\[\tcbparskip]

\textbf{\ul{NOTE}: This should not be done for other branches or for personal versions of code}.
\end{tcbparbox}
%-----------------------------------------------------------------------------------------------------------------------------------------------------------------------%
%------------------------------------ ion/pCT_code/<Preprocessing/Reconstruction>/<username>/... --------------------------------%
%-----------------------------------------------------------------------------------------------------------------------------------------------------------------------%
\begin{tcbparbox}|\dirsep ion\dirsep pCT\_code\dirsep user\_code\dirsep\caratenclosed*{username}\dirsep$\dots$|
\bfdash Contains subdirectories for each pCT user where they can store and modify clones of the pCT program repositories and their personal code.
\end{tcbparbox}
%-----------------------------------------------------------------------------------------------------------------------------------------------------------------------%
\end{tcbenvironment}
%%%%%%%%%%%%%%%%%%%%%%%%%%%%%%%%%%%%%%%%%%%%%%%%%%%%%%%%%%%%%%%%%%%%%%%%%%%%%%%%%%%%%%%%%%%%%%%%%%%%%%%%%%%%%%%%%%%%%%%%%%%%%%%%%%%%%%%%%%%%%%%%%
%%%%%%%%%%%%%%%%%%%%%%%%%%%%%%%%%%%%%%%%%%%%%%%%%%%%%%%%%%%%%%%%%%%%%%%%%%%%%%%%%%%%%%%%%%%%%%%%%%%%%%%%%%%%%%%%%%%%%%%%%%%%%%%%%%%%%%%%%%%%%%%%%
%%%%%%%%%%%%%%%%%%%%%%%%%%%%%%%%%%%%%%%%%%%%%%%%%%%%%%%%%%%%%%%%%%%%%%%%%%%%%%%%%%%%%%%%%%%%%%%%%%%%%%%%%%%%%%%%%%%%%%%%%%%%%%%%%%%%%%%%%%%%%%%%%
\endinput 
\idxTestContent
\glsTestContent
\abbrevTestContent
%-------------------------------------------------------------------------------------------------------------------------------%
%\clearpage%
\Section*{Tardis Directories}[Description and purpose/usage of the private user and shared directories on the Tardis compute nodes]
%-------------------------------------------------------------------------------------------------------------------------------%
%-------------------------------------------------------------------------------------------------------------------------------%
%-------------------------------------------------------------------------------------------------------------------------------%
\begin{tcbenvironment}|Tardis|%
%-------------------------------------------------------------------------------------------------------------------------------%
%-------------------------------------------------------------------------------------------------------------------------------%
%-------------------------------------------------------------------------------------------------------------------------------%
\begin{tcbparbox}|\dirsep local\dirsep$\dots$|%
    \bfdash This is the parent directory for all pCT code and data on each Tardis compute node's local solid-state drive, the equivalent of the \docentry[constdir]{ion} directory on Kodiak's network attached storage device.  This data is stored on the compute nodes' local solid-state drive and is not backed up, so users must ensure they copy data to Kodiak if they want to store it permanently
\end{tcbparbox}
%%%%%%%%%%%%%%%%%%%%%%%%%%%%%%%%%%%%%%%%%%%%%%%%%%%%%%%%%%%%%%%%%%%%%%%%%%%%%%%%%%%%%%%%%%%%%%%%%%%%%%%%%%%%%%%%%%%%%%%%%%%%%%%%%%%%%%%%%%%%%%%%%
%%%%%%%%%%%%%%%%%%%%%%%%%%%%%%%%%%%%% ion/pCT_code/... %%%%%%%%%%%%%%%%%%%%%%%%%%%%%%%%%%%%%%%%%%%
%%%%%%%%%%%%%%%%%%%%%%%%%%%%%%%%%%%%%%%%%%%%%%%%%%%%%%%%%%%%%%%%%%%%%%%%%%%%%%%%%%%%%%%%%%%%%%%%%%%%%%%%%%%%%%%%%%%%%%%%%%%%%%%%%%%%%%%%%%%%%%%%%
\begin{tcbparbox}|\dirsep local\dirsep pCT\_code\dirsep $\dots$|
\bfdash This directory is used to store pCT code on the Tardis compute nodes and corresponds to the \docentry[constdir]{ion\dirsep pCT\_code} directory on Kodiak (with the same organizatioal scheme as well).
\end{tcbparbox}
%%%%%%%%%%%%%%%%%%%%%%%%%%%%%%%%%%%%%%%%%%%%%%%%%%%%%%%%%%%%%%%%%%%%%%%%%%%%%%%%%%%%%%%%%%%%%%%%%%%%%%%%%%%%%%%%%%%%%%%%%%%%%%%%%%%%%%%%%%%%%%%%%
%%%%%%%%%%%%%%%% ion/pCT_code/<Preprocessing/Reconstruction>/<username>/... %%%%%%%%%%%%%%%%%%%%%%%
%%%%%%%%%%%%%%%%%%%%%%%%%%%%%%%%%%%%%%%%%%%%%%%%%%%%%%%%%%%%%%%%%%%%%%%%%%%%%%%%%%%%%%%%%%%%%%%%%%%%%%%%%%%%%%%%%%%%%%%%%%%%%%%%%%%%%%%%%%%%%%%%%
\begin{tcbparbox}%
     |\dirsep local\dirsep pCT\_code\dirsep user\_code\dirsep\caratenclosed*{username}\dirsep$\dots$|
\bfdash The equivalent of Kodiak's \docentry[constdir]{ion\dirsep pCT\_code\dirsep user\_code\dirsep\caratenclosed*{username}\dirsep$\dots$} subdirectories where pCT users can copy/modify/execute their personal clones of pCT program repositories and their personal code on Tardis compute nodes
\end{tcbparbox}
%%%%%%%%%%%%%%%%%%%%%%%%%%%%%%%%%%%%%%%%%%%%%%%%%%%%%%%%%%%%%%%%%%%%%%%%%%%%%%%%%%%%%%%%%%%%%%%%%%%%%%%%%%%%%%%%%%%%%%%%%%%%%%%%%%%%%%%%%%%%%%%%%
%%%%%%%%%%%%%%%%%%%% ion/pCT_code/git/<GitHub account>/<GitHub repo>/... %%%%%%%%%%%%%%%%%%%%%%%%%%
%%%%%%%%%%%%%%%%%%%%%%%%%%%%%%%%%%%%%%%%%%%%%%%%%%%%%%%%%%%%%%%%%%%%%%%%%%%%%%%%%%%%%%%%%%%%%%%%%%%%%%%%%%%%%%%%%%%%%%%%%%%%%%%%%%%%%%%%%%%%%%%%%
\begin{tcbparbox}|\dirsep local\dirsep pCT\_code\dirsep git\dirsep\caratenclosed*{GitHub account}\dirsep\caratenclosed*{GitHub repository}\dirsep $\dots$|
\bfdash This directory contains clones of the available pCT GitHub accounts and repositories, with parent directories for each GitHub account and subdirectories for each of their repositories.  Each program repository has a \docentry[gitbranch]{master} branch, which typically corresponds to the current release version (though there may also be a branch like \docentry[gitbranch]{release} used instead) and each of the program's developers will typically have their own branch which they can use to develop and test new ideas/features.  The group of developers of a program should decide amongst themselves what the process will be for approving merges with the \docentry[gitbranch]{master/release} branch and when to release a new version of the program, which may include the results of several separate merges.\\\par

Users accessing the \docentry[gitbranch]{master/release} branch of these clones should execute \tcbinlinebashbox!{git pull --rebase} prior to using the code to ensure it is updated to its latest version.\\\par

\textbf{\ul{NOTE}: This should not be done for other branches or the personal versions of code}.
\end{tcbparbox}
%%%%%%%%%%%%%%%%%%%%%%%%%%%%%%%%%%%%%%%%%%%%%%%%%%%%%%%%%%%%%%%%%%%%%%%%%%%%%%%%%%%%%%%%%%%%%%%%%%%%%%%%%%%%%%%%%%%%%%%%%%%%%%%%%%%%%%%%%%%%%%%%%
%%%%%%%%%%%%%%%%%%%%%%%%%%%%%%%%%%%%% ion/pCT_data/... %%%%%%%%%%%%%%%%%%%%%%%%%%%%%%%%%%%%%%%%%%
%%%%%%%%%%%%%%%%%%%%%%%%%%%%%%%%%%%%%%%%%%%%%%%%%%%%%%%%%%%%%%%%%%%%%%%%%%%%%%%%%%%%%%%%%%%%%%%%%%%%%%%%%%%%%%%%%%%%%%%%%%%%%%%%%%%%%%%%%%%%%%%%%
\begin{tcbparbox}|\dirsep local\dirsep pCT\_data\dirsep$\dots$|
\bfdash This directory is where pCT data is to be copied from Kodiak and corresponds to the \docentry[constdir]{ion\dirsep pCT\_data} directory on Kodiak.
\end{tcbparbox}
%%%%%%%%%%%%%%%%%%%%%%%%%%%%%%%%%%%%%%%%%%%%%%%%%%%%%%%%%%%%%%%%%%%%%%%%%%%%%%%%%%%%%%%%%%%%%%%%%%%%%%%%%%%%%%%%%%%%%%%%%%%%%%%%%%%%%%%%%%%%%%%%%
%%%%%%%%%%%%%%%%%%%%%%%%%%%%%%%%%%%%% ion/pCT_data/... %%%%%%%%%%%%%%%%%%%%%%%%%%%%%%%%%%%%%%%%%%
%%%%%%%%%%%%%%%%%%%%%%%%%%%%%%%%%%%%%%%%%%%%%%%%%%%%%%%%%%%%%%%%%%%%%%%%%%%%%%%%%%%%%%%%%%%%%%%%%%%%%%%%%%%%%%%%%%%%%%%%%%%%%%%%%%%%%%%%%%%%%%%%%
\begin{tcbparbox}|\dirsep local\dirsep pCT\_data\dirsep user\_data\dirsep\caratenclosed*{username}\dirsep$\dots$|
\bfdash Subdirectories for each pCT user on each Tardis compute node where they can transfer data they want to reconstruct when the data is not organized according to the pCT data naming/organizational scheme.  If the output data they generate will also be unorganized, it should also be written to this directory.\\\par

\textbf{\ul{NOTE}: The \docentry[gitrepo]{pCT-collaboration/pCT\_Tools} repository contains a script which loads a number of bash functions useful to pCT users, including a function that can be used to organize and then copy unorganized data to the organized data directory on the Tardis compute nodes}
\end{tcbparbox}
%%%%%%%%%%%%%%%%%%%%%%%%%%%%%%%%%%%%%%%%%%%%%%%%%%%%%%%%%%%%%%%%%%%%%%%%%%%%%%%%%%%%%%%%%%%%%%%%%%%%%%%%%%%%%%%%%%%%%%%%%%%%%%%%%%%%%%%%%%%%%%%%%
%%%%%%%%%%%%%%%%%%%%%%%%%%%%%%%%%%%%%%%%%%%%%%%%%%%%%%%%%%%%%%%%%%%%%%%%%%%%%%%%%%%%%%%%%%%%%%%%%%%%%%%%%%%%%%%%%%%%%%%%%%%%%%%%%%%%%%%%%%%%%%%%%
%%%%%%%%%%%%%%%%%%%%%%%%%%%%%%%%%%%%%%%%%%%%%%%%%%%%%%%%%%%%%%%%%%%%%%%%%%%%%%%%%%%%%%%%%%%%%%%%%%%%%%%%%%%%%%%%%%%%%%%%%%%%%%%%%%%%%%%%%%%%%%%%%
\end{tcbenvironment}
%%%%%%%%%%%%%%%%%%%%%%%%%%%%%%%%%%%%%%%%%%%%%%%%%%%%%%%%%%%%%%%%%%%%%%%%%%%%%%%%%%%%%%%%%%%%%%%%%%%%%%%%%%%%%%%%%%%%%%%%%%%%%%%%%%%%%%%%%%%%%%%%%
%%%%%%%%%%%%%%%%%%%%%%%%%%%%%%%%%%%%%%%%%%%%%%%%%%%%%%%%%%%%%%%%%%%%%%%%%%%%%%%%%%%%%%%%%%%%%%%%%%%%%%%%%%%%%%%%%%%%%%%%%%%%%%%%%%%%%%%%%%%%%%%%%
%%%%%%%%%%%%%%%%%%%%%%%%%%%%%%%%%%%%%%%%%%%%%%%%%%%%%%%%%%%%%%%%%%%%%%%%%%%%%%%%%%%%%%%%%%%%%%%%%%%%%%%%%%%%%%%%%%%%%%%%%%%%%%%%%%%%%%%%%%%%%%%%%
\endinput 
%*******************************************************************************************************************************%
%*******************************************************************************************************************************%
%*******************************************************************************************************************************%
\Chapter{Organizational Scheme}
%\Section{Organization}%[Description of directory hierarchies relevant to storage of code (see \autohyperlink{]
Now this is in the middle of a paragrap
\glsTestContent
\idxTestContent
\abbrevTestContent
%-------------------------------------------------------------------------------------------------------------------------------%
%-------------------------------------------------------------------------------------------------------------------------------%
%-------------------------------------------------------------------------------------------------------------------------------%
%\Section!{Code Organization}%[Description of directory hierarchies relevant to storage of code (see \autohyperlink{]
\Section{Code Organization}%[Description of directory hierarchies relevant to storage of code (see \autohyperlink{]
%-------------------------------------------------------------------------------------------------------------------------------%
%\clearpage%
\Subsection(pCT Code Hierarchy){\texttt{pCT\_Code} Hierarchy}
%%%%%%%%%%%%%%%%%%%%%%%%%%%%%%%%%%%%%%%%%%%%%%%%%%%%%%%%%%%%%%%%%%%%%%%%%%%%%%%%%%%%%%%%%%%%%%%%%%%%%%%%%%%%%%%%%%%%%%%%%%%%%%%%%%%%%%%%%%%%%%%%%
%%%%%%%%%%%%%%%%%%%%%%%%%%%%%%%%%%%%%%%%%%%%%%%%%%%%%%%%%%%%%%%%%%%%%%%%%%%%%%%%%%%%%%%%%%%%%%%%%%%%%%%%%%%%%%%%%%%%%%%%%%%%%%%%%%%%%%%%%%%%%%%%%
%%%%%%%%%%%%%%%%%%%%%%%%%%%%%%%%%%%%%%%%%%%%%%%%%%%%%%%%%%%%%%%%%%%%%%%%%%%%%%%%%%%%%%%%%%%%%%%%%%%%%%%%%%%%%%%%%%%%%%%%%%%%%%%%%%%%%%%%%%%%%%%%%
\begin{tcbenvironment}|pCT\_code|
%%%%%%%%%%%%%%%%%%%%%%%%%%%%%%%%%%%%%%%%%%%%%%%%%%%%%%%%%%%%%%%%%%%%%%%%%%%%%%%%%%%%%%%%%%%%%%%%%%%%%%%%%%%%%%%%%%%%%%%%%%%%%%%%%%%%%%%%%%%%%%%%%
\begin{tcbparbox}|\dirsep ion/local\dirsep pCT\_code\dirsep$\dots$|
\bfdash directory on Kodiak and each Tardis compute node used to store clones of the GitHub repositories relevant to pCT and the private storage of user code.  The naming and organizational scheme is the same on Kodiak and each Tardis compute node, thereby simplifying distribution of code for execution on Tardis; the only difference is the top level parent directory on Kodiak is \docentry[constdir]{ion} and on the Tardis compute nodes it is \docentry[constdir]{local}, but their subdirectories are identical.  See the \autohyperlink[type=Section]{GitHub Accounts/Repositories} section for a list of GitHub accounts and repositories relevant to pCT.  For a visual representation of this hierarchy, see the \autohyperlink[type=Section, label=pCT\_code Hierarchy Diagram]{pCT code Hierarchy Diagram}.
\end{tcbparbox}
%%%%%%%%%%%%%%%%%%%%%%%%%%%%%%%%%%%%%%%%%%%%%%%%%%%%%%%%%%%%%%%%%%%%%%%%%%%%%%%%%%%%%%%%%%%%%%%%%%%%%%%%%%%%%%%%%%%%%%%%%%%%%%%%%%%%%%%%%%%%%%%%%
%\begin{tcbparbox}{tcbenumeratedStyle}
\begin{tcbenumbox}
\begin{ThinEnum}[labelindent=1pt, leftmargin=*]
    \item \docentry[constdir]{ion} or \docentry[constdir]{local} : parent directory for all pCT code/data on Kodiak and the Tardis compute nodes
        \begin{ThinEnum}[labelindent=1pt, leftmargin=*]
            \item \docentry[constdir]{pCT\_code} : directory containing the code for all pCT programs linked to their GitHub repositories as well as subdirectories for each pCT user where they can clone and modify these repositories and store/execute their own code
            \begin{ThinEnum}[labelindent=1pt, leftmargin=*]
                \item \docentry[constdir]{git} : directory containing clones of the standard/common pCT programs, providing easy and immediate access to the newest version of each of these programs
                \begin{ThinEnum}[labelindent=1pt, leftmargin=*]
                    \item \docentry[dir]{\caratenclosed*{GitHub account}} : directories for each of the GitHub accounts containing one or more pCT programs
                	\begin{ThinEnum}[labelindent=1pt, leftmargin=*]
                        \item \docentry[dir]{\caratenclosed*{GitHub repository}} : subdirectories for each pCT code repository in the associated GitHub account
                    \end{ThinEnum}
                \end{ThinEnum}
            \item \docentry[constdir]{user\_code} : directory containing subdirectories for each user where they can store their personal code
                \begin{ThinEnum}[labelindent=1pt, leftmargin=*]
                    \item \docentry[dir]{\caratenclosed*{username}} : subdirectories for each pCT user where they can store their personal code
                \end{ThinEnum}
        \end{ThinEnum}
        \end{ThinEnum}
\end{ThinEnum}
\end{tcbenumbox}
%\end{tcbparbox}
\end{tcbenvironment}
\endinput
%\clearpage

%-------------------------------------------------------------------------------------------------------------------------------%
%-------------------------------------------------------------------------------------------------------------------------------%
%-------------------------------------------------------------------------------------------------------------------------------%
%\clearpage%
\Section{Data Organization}
%\Section*!{Data Organization}
%-------------------------------------------------------------------------------------------------------------------------------%
%\clearpage%
\Subsection(Organized Data Hierarchy){\texttt{organized\_data} Hierarchy}
%%%%%%%%%%%%%%%%%%%%%%%%%%%%%%%%%%%%%%%%%%%%%%%%%%%%%%%%%%%%%%%%%%%%%%%%%%%%%%%%%%%%%%%%%%%%%%%%%%%%%%%%%%%%%%%%%%%%%%%%%%%%%%%%%%%%%%%%%%%%%%%%%
%%%%%%%%%%%%%%%%%%%%%%%%%%%%%%%%%%%%%%%%%%%%%%%%%%%%%%%%%%%%%%%%%%%%%%%%%%%%%%%%%%%%%%%%%%%%%%%%%%%%%%%%%%%%%%%%%%%%%%%%%%%%%%%%%%%%%%%%%%%%%%%%%
%%%%%%%%%%%%%%%%%%%%%%%%%%%%%%%%%%%%%%%%%%%%%%%%%%%%%%%%%%%%%%%%%%%%%%%%%%%%%%%%%%%%%%%%%%%%%%%%%%%%%%%%%%%%%%%%%%%%%%%%%%%%%%%%%%%%%%%%%%%%%%%%%
\begin{tcbenvironment}|organized\_data|
%%%%%%%%%%%%%%%%%%%%%%%%%%%%%%%%%%%%%%%%%%%%%%%%%%%%%%%%%%%%%%%%%%%%%%%%%%%%%%%%%%%%%%%%%%%%%%%%%%%%%%%%%%%%%%%%%%%%%%%%%%%%%%%%%%%%%%%%%%%%%%%%%
\begin{tcbparbox}|\dirsep ion\dirsep pCT\_data\dirsep organized\_data|%
\bfdash directory containing all raw, preprocessed/projection, and reconstruction data files, primarily soft symbolic links to the actual data stored elsewhere by data type, organized into a hierarchy of directories indicating data dependencies.  A visual representation of this hierarchy can be seen in the \autohyperlink[type=Section, label=organized\_data Hierarchy Diagram]{Organized Data Hierarchy Diagram}. Reconstruction can optionally generate a number of additional data files containing intermediate data useful in debugging and analysis, but only the default data/image files are shown here for brevity (see the \autohyperlink[type=Section]{Reconstruction File List} section for a full list of reconstruction data/image files).
%
%Only the default reconstruction data/image files are shown here, for a full list of additional files which can optionally be written to disk, see the \autohyperlink[type=Section]{Reconstruction File List} section.
\end{tcbparbox}
%%%%%%%%%%%%%%%%%%%%%%%%%%%%%%%%%%%%%%%%%%%%%%%%%%%%%%%%%%%%%%%%%%%%%%%%%%%%%%%%%%%%%%%%%%%%%%%%%%%%%%%%%%%%%%%%%%%%%%%%%%%%%%%%%%%%%%%%%%%%%%%%%
%\begin{tcbparbox}{tcbenumeratedStyle}
\begin{tcbenumbox}
    \begin{ThinEnum}
        \item \docentry[dir]{Phantom} : directory containing all of the experimental/simulated data and reconstructed images associated with this phantom/object.
        \begin{ThinEnum}
            \item \docentry[constdir]{Reference\_Images} : directory containing reference images (xCT, RSP, etc) relevant to analysis/comparison of the data/images for this object and data type.
            \item \docentry[constdir]{Experimental} : directory containing data and images generated from experimental scans of the object.
            \begin{ThinEnum}
                \item \docentry[dir]{YY-MM-DD} : directory containing data and reconstructed images corresponding to the experimental scan of the object performed on this date.
                \begin{ThinEnum}
                    \item \docentry[dir]{XXXX[\_AAA]} : directory containing data/images corresponding to the 4-digit run \# \docentry![subcategory]{XXXX}, potentially including \emph{``subcategory tag(s)''} of the form \docentry![subcategory]{\_AAA} indicating, e.g., a continuous scan (\docentry![subcategory]{\_Cont}), phantom position/section (inferior (\docentry![subcategory]{\_Inf}) or superior (\docentry![subcategory]{\_Sup}), top (\docentry![subcategory]{\_Top}) or bottom (\docentry![subcategory]{\_Bot}), etc.).
%************************************************************************************************************************************************%
                    \begin{ThinEnum}
                        \item \docentry[constdir]{Input} : directory containing raw data generated by object scan from each gantry angle and transmitted by event builder.
                        \begin{ThinEnum}
                            \item \docentry[fileset]{raw\_xxx.bin} : binary files containing trigger/tracker/energy detector data from event builder associated with gantry angle \docentry![subcategory]{xxx} =\{\docentry![subcategory]{001}, \docentry![subcategory]{002}, \docentry![subcategory]{003}, $\cdots\}$.
                        \end{ThinEnum}%\clearpage
                        \item \docentry[constdir]{Output} : directory containing calibration and post processed data generated from analysis of raw data and used as input to image reconstruction.
                        \begin{ThinEnum}
                            \item \docentry[dir]{YY-MM-DD} : directory containing the post processed \docentry![fileset]{projection\_xxx.bin} data generated on this date and the reconstructions using this data.
                    \begin{ThinEnum}
                                \item \docentry[file]{readme.txt} : contains input raw data info, phantom name, and run date.
                                \item \docentry[file]{TVcorr.txt} : contains TV corrected WEPL calibration curve coefficients.
                                \item \docentry[file]{WcalibTemp.txt} : temporary file containing WEPL calibration curve coefficients.
                                    \item \docentry[file]{Wcalib.txt} : contains final WEPL calibration curve coefficients.
                                \item \docentry[fileset]{projection\_xxx.bin} : preprocessed data files containing tracker coordinates and WEPL data for gantry angle $\text{\docentry![subcategory]{xxx}} =\{\text{\docentry![subcategory]{001}}, \text{\docentry![subcategory]{002}}, \text{\docentry![subcategory]{003}}, \cdots\}$ used as input to image reconstruction.
%************************************************************************************************************************************************%
                                \item \docentry[constdir]{Reconstruction} : directory containing preprocessed data and reconstructed
                                        images generated using the \docentry![fileset]{projection\_xxx.bin} data along with reference images relevant to the object.
                                    \begin{ThinEnum}
                                        \item \docentry[file]{settings.cfg} : configuration file containing key/value pairs specifying scan/phantom properties
                                    (phantom, run date/\#/tag(s), etc.) and default reconstruction settings/parameters.
                                    \item \docentry[dir]{YY-MM-DD} : directory containing the preprocessed data generated on this date and the reconstructed images generated from this data.
                                    \begin{ThinEnum}
                                            \item \docentry[file]{execution\_log.txt} : execution times for various portions of preprocessing and/or reconstruction and total program execution time.
                                            \item \docentry[file]{FBP.txt} : text image of filtered back projection (FBP) image.
                                            \item \docentry[file]{FBP.png} : conversion of \docentry![file]{FBP.txt} to PNG image.
                                            \item \docentry[file]{FBP\_med\_filtered.txt} : text image result of applying median filter to the filtered back projection (FBP) image.
                                            \item \docentry[file]{FBP\_med\_filtered.png} : conversion of \docentry![file]{FBP\_avg\_filtered.txt} to PNG image.
                                            \item \docentry[file]{hull.txt} : text image of selected object hull in 1s/0s.
                                            \item \docentry[file]{hull.png} : conversion of \docentry![file]{hull.txt} to PNG image.
                                            \item \docentry[file]{settings\_log.cfg} : copy of \docentry![file]{settings.cfg} with any changes made to parameters/options applied at execution, if any.
                                            \item \docentry[file]{TV\_measurements.txt} : total variation (TV) measurements before/after each iteration
                                            \item \docentry[file]{x\_0.txt} : text image of initial iterate.
                                            \item \docentry[file]{x\_0.png} : conversion of \docentry![file]{x\_0.txt} to PNG image.
                                        \item \docentry[constdir]{Images} : directory containing reconstructed images from the preprocessed data.
                                        \begin{ThinEnum}
                                            \item \docentry[dir]{YY-MM-DD} : directory containing reconstructed images generated from the preprocessed data on this date.
                                                \begin{ThinEnum}
                                                    \item \docentry[fileset]{x\_k.txt} : text image of reconstructed image $\boldsymbol{x^k}$ after $\boldsymbol{k}$ iterations.
                                                    \item \docentry[fileset]{x\_k.png} : PNG image of reconstructed image $\boldsymbol{x^k}$ after $\boldsymbol{k}$ iterations.
                                                \end{ThinEnum}%
                                        \end{ThinEnum}%
                                    \end{ThinEnum}%
                                \end{ThinEnum}%
                            \end{ThinEnum}%
                        \end{ThinEnum}%
                    \end{ThinEnum}%
                \end{ThinEnum}%
            \end{ThinEnum}%
            \item \docentry[constdir]{Simulated} : directory containing data and images generated from simulated scans of the object.
            \begin{ThinEnum}
                    \item \docentry[dir]{G\_YY-MM-DD} : directory containing data and reconstructed images from the GEANT4 simulated scan of the object generated on this date..
            \begin{ThinEnum}
                \item \docentry[dir]{XXXX[\_AAA]} : directory containing data/images corresponding to the 4-digit run \# \docentry![subcategory]{XXXX}, potentially including \emph{``subcategory tag(s)''} of the form \docentry![subcategory]{\_AAA} indicating, e.g., a continuous scan (\docentry![subcategory]{\_Cont}), phantom position/section (inferior (\docentry![subcategory]{\_Inf}) or superior (\docentry![subcategory]{\_Sup}), top (\docentry![subcategory]{\_Top}) or bottom (\docentry![subcategory]{\_Bot}), etc.).
%*****************************************************************************************************************************%
                   \begin{ThinEnum}
                        \item \docentry[constdir]{Input} : directory containing simulated raw data files for each gantry angle.
                        \begin{ThinEnum}
                            \item \docentry[fileset]{raw\_xxx.bin} : binary files containing trigger/tracker/energy detector data from event builder associated with gantry angle $\text{\docentry![subcategory]{xxx}} =\{\text{\docentry![subcategory]{001}}, \text{\docentry![subcategory]{002}}, \text{\docentry![subcategory]{003}}, \cdots\}$.
                        \end{ThinEnum}
                        \item \docentry[constdir]{Output} : directory containing calibration and post processed data generated from analysis of raw data and used as input to image reconstruction.
                        \begin{ThinEnum}
                                \item \docentry[dir]{YY-MM-DD} : directory containing the post processed \docentry![fileset]{projection\_xxx.bin} data generated on this date and the reconstructions using this data.
                                \begin{ThinEnum}
                                    \item \docentry[file]{readme.txt} : contains input raw data info, phantom name, and run date.
                                    \item \docentry[file]{TVcorr.txt} : contains TV corrected WEPL calibration curve coefficients.
                                    \item \docentry[file]{WcalibTemp.txt} : temporary file containing WEPL calibration curve coefficients.
                                    \item \docentry[file]{Wcalib.txt} : contains final WEPL calibration curve coefficients.
                                    \item \docentry[fileset]{projection\_xxx.bin} : preprocessed data files containing tracker coordinates and WEPL data for gantry angle $\text{\docentry![subcategory]{xxx}} =\{\text{\docentry![subcategory]{001}}, \text{\docentry![subcategory]{002}}, \text{\docentry![subcategory]{003}}, \cdots\}$ used as input to image reconstruction.
%************************************************************************************************************************************************%
                                    \item \docentry[constdir]{Reconstruction} : directory containing preprocessed data and reconstructed images generated using the \docentry![fileset]{projection\_xxx.bin} data along with reference images relevant to the object.
                               \begin{ThinEnum}
                                    \item \docentry[file]{settings.cfg} : configuration file containing key/value pairs specifying scan/phantom properties
                                    (phantom, run date/\#/tag(s), etc.) and default reconstruction settings/parameters.
                                    \item \docentry[dir]{YY-MM-DD} : directory containing the preprocessed data generated on this date and the reconstructed images generated from this data.
                                    \begin{ThinEnum}
                                            \item \docentry[file]{execution\_log.txt} : execution times for various portions of preprocessing and/or reconstruction and total program execution time.
                                            \item \docentry[file]{FBP.txt} : text image of filtered back projection (FBP) image.
                                            \item \docentry[file]{FBP.png} : conversion of \docentry![file]{FBP.txt} to PNG image.
                                            \item \docentry[file]{FBP\_med\_filtered.txt} : text image result of applying median filter to the filtered back projection (FBP) image.
                                            \item \docentry[file]{FBP\_med\_filtered.png} : conversion of \docentry![file]{FBP\_avg\_filtered.txt} to PNG image.
                                            \item \docentry[file]{hull.txt} : text image of selected object hull in 1s/0s.
                                            \item \docentry[file]{hull.png} : conversion of \docentry![file]{hull.txt} to PNG image.
                                            \item \docentry[file]{settings\_log.cfg} : copy of \docentry![file]{settings.cfg} with any changes made to parameters/options applied at execution, if any.
                                            \item \docentry[file]{TV\_measurements.txt} : total variation (TV) measurements before/after each iteration
                                            \item \docentry[file]{x\_0.txt} : text image of initial iterate.
                                            \item \docentry[file]{x\_0.png} : conversion of \docentry![file]{x\_0.txt} to PNG image.
                                            \item \docentry[constdir]{Images} : directory containing reconstructed images from the preprocessed data.
                                        \begin{ThinEnum}
                                            \item \docentry[dir]{YY-MM-DD} : directory containing reconstructed images generated from the preprocessed data on this date.
                                                \begin{ThinEnum}
                                                    \item \docentry[fileset]{x\_k.txt} : text image of reconstructed image $\boldsymbol{x^k}$ after $\boldsymbol{k}$ iterations.
                                                    \item \docentry[fileset]{x\_k.png} : PNG image of reconstructed image $\boldsymbol{x^k}$ after $\boldsymbol{k}$ iterations.
                                                \end{ThinEnum}%
                                        \end{ThinEnum}%
                                    \end{ThinEnum}%
                                \end{ThinEnum}%
                            \end{ThinEnum}%
                        \end{ThinEnum}%
                    \end{ThinEnum}%
                \end{ThinEnum}%
                \item \docentry[dir]{T\_YY-MM-DD} : directory containing data and reconstructed images corresponding to all TOPAS simulated scans of the object generated on this date.
                \begin{ThinEnum}
                \item \docentry[dir]{XXXX[\_AAA]} : directory containing data/images corresponding to the 4-digit run \# \docentry![subcategory]{XXXX}, potentially including \emph{``subcategory tag(s)''} of the form \docentry![subcategory]{\_AAA} indicating, e.g., a continuous scan (\docentry![subcategory]{\_Cont}), phantom position/section (inferior (\docentry![subcategory]{\_Inf}) or superior (\docentry![subcategory]{\_Sup}), top (\docentry![subcategory]{\_Top}) or bottom (\docentry![subcategory]{\_Bot}), etc.).
                    \begin{ThinEnum}
%************************************************************************************************************************************************%
                        \item \docentry[constdir]{Input} : directory containing simulated raw data files generated for each gantry angle.
                        \begin{ThinEnum}
                            \item \docentry[fileset]{raw\_xxx.bin} : binary files containing trigger/tracker/energy detector data from event builder associated with gantry angle $\text{\docentry[subcategory]{xxx}} =\{\text{\docentry[subcategory]{001}}, \text{\docentry![subcategory]{002}}, \text{\docentry[subcategory]{003}}, \cdots\}$.
                        \end{ThinEnum}
                        \item \docentry[constdir]{Output} : directory containing calibration and post processed data generated from analysis of raw data and used as input to image reconstruction.
                        \begin{ThinEnum}
                            \item \docentry[dir]{YY-MM-DD} : directory containing the post processed \docentry![fileset]{projection\_xxx.bin} data generated on this date and the reconstructions using this data.
                            \begin{ThinEnum}
                                \item \docentry[file]{readme.txt} : contains input raw data info, phantom name, and run date.
                                \item \docentry[file]{TVcorr.txt} : contains TV corrected WEPL calibration curve coefficients.
                                \item \docentry[file]{WcalibTemp.txt} : temporary file containing WEPL calibration curve coefficients.
                                    \item \docentry[file]{Wcalib.txt} : contains final WEPL calibration curve coefficients.
                                \item \docentry[fileset]{projection\_xxx.bin} : preprocessed data files containing tracker coordinates and WEPL data for gantry angle $\text{\docentry![subcategory]{xxx}} =\{\text{\docentry![subcategory]{001}}, \text{\docentry![subcategory]{002}}, \text{\docentry![subcategory]{003}}, \cdots\}$ used as input to image reconstruction.
%************************************************************************************************************************************************%
%************************************************************************************************************************************************%
                                \item \docentry[constdir]{Reconstruction} : directory containing preprocessed data and reconstructed images generated using the \docentry![fileset]{projection\_xxx.bin} data along with reference images relevant to the object.
                               \begin{ThinEnum}
                                    \item \docentry[file]{settings.cfg} : configuration file containing key/value pairs specifying scan/phantom properties
                                    (phantom, run date/\#/tag(s), etc.) and default reconstruction settings/parameters.
                                    \item \docentry[dir]{YY-MM-DD} : directory containing the pre-reconstruction processed data generated on this date and the reconstructed images generated from this data.
                                    \begin{ThinEnum}
                                            \item \docentry[file]{execution\_log.txt} : execution times for various portions of preprocessing and/or reconstruction and total program execution time.
                                            \item \docentry[file]{FBP.txt} : text image of filtered back projection (FBP) image.
                                            \item \docentry[file]{FBP.png} : conversion of \docentry![file]{FBP.txt} to PNG image.
                                            \item \docentry[file]{FBP\_med\_filtered.txt} : text image result of applying median filter to the filtered back projection (FBP) image.
                                            \item \docentry[file]{FBP\_med\_filtered.png} : conversion of \docentry![file]{FBP\_avg\_filtered.txt} to PNG image.
                                            \item \docentry[file]{hull.txt} : text image of selected object hull in 1s/0s.
                                            \item \docentry[file]{hull.png} : conversion of \docentry![file]{hull.txt} to PNG image.
                                            \item \docentry[file]{settings\_log.cfg} : copy of \docentry![file]{settings.cfg} with any changes made to parameters/options applied at execution, if any.
                                            \item \docentry[file]{TV\_measurements.txt} : total variation (TV) measurements before/after each iteration
                                            \item \docentry[file]{x\_0.txt} : text image of initial iterate.
                                            \item \docentry[file]{x\_0.png} : conversion of \docentry![file]{x\_0.txt} to PNG image.
                                            \item \docentry[constdir]{Images} : directory containing reconstructed images from the preprocessed data.
                                            \begin{ThinEnum}
                                            \item \docentry[dir]{YY-MM-DD} : directory containing reconstructed images generated from the preprocessed data on this date.
                                                \begin{ThinEnum}
                                                    \item \docentry[fileset]{x\_k.txt} : text image of reconstructed image $\boldsymbol{x^k}$ after $\boldsymbol{k}$ iterations.
                                                    \item \docentry[fileset]{x\_k.png} : PNG image of reconstructed image $\boldsymbol{x^k}$ after $\boldsymbol{k}$ iterations.
                                                \end{ThinEnum}%
                                        \end{ThinEnum}%
                                    \end{ThinEnum}%
                                \end{ThinEnum}%
                            \end{ThinEnum}%
                        \end{ThinEnum}%
                    \end{ThinEnum}%
                \end{ThinEnum}%
            \end{ThinEnum}%
        \end{ThinEnum}%
    \end{ThinEnum}%
%\end{tcbparbox}%
\end{tcbenumbox}
%%%%%%%%%%%%%%%%%%%%%%%%%%%%%%%%%%%%%%%%%%%%%%%%%%%%%%%%%%%%%%%%%%%%%%%%%%%%%%%%%%%%%%%%%%%%%%%%%%%%%%%%%%%%%%%%%%%%%%%%%%%%%%%%%%%%%%%%%%%%%%%%%
\end{tcbenvironment}
\endinput
%%%%%%%%%%%%%%%%%%%%%%%%%%%%%%%%%%%%%%%%%%%%%%%%%%%%%%%%%%%%%%%%%%%%%%%%%%%%%%%%%%%%%%%%%%%%%%%%%%%%%%%%%%%%%%%%%%%%%%%%%%%%%%%%%%%%%%%%%%%%%%%%%
%\newpage
%%%%%%%%%%%%%%%%%%%%%%%%%%%%%%%%%%%%%%%%%%%%%%%%%%%%%%%%%%%%%%%%%%%%%%%%%%%%%%%%%%%%%%%%%%%%%%%%%%%%%%%%%%%%%%%%%%%%%%%%%%%%%%%%%%%%%%%%%%%%%%%%%

%\clearpage%
%\par\bigskip\bigskip%
\Subsection(Raw Data Hierarchy){\texttt{raw\_data} Hierarchy}
%\Subsection*(Raw Data Hierarchy){\texttt{raw\_data} Hierarchy}
%%%%%%%%%%%%%%%%%%%%%%%%%%%%%%%%%%%%%%%%%%%%%%%%%%%%%%%%%%%%%%%%%%%%%%%%%%%%%%%%%%%%%%%%%%%%%%%%%%%%%%%%%%%%%%%%%%%%%%%%%%%%%%%%%%%%%%%%%%%%%%%%%
%%%%%%%%%%%%%%%%%%%%%%%%%%%%%%%%%%%%%%%%%%%%%%%%%%%%%%%%%%%%%%%%%%%%%%%%%%%%%%%%%%%%%%%%%%%%%%%%%%%%%%%%%%%%%%%%%%%%%%%%%%%%%%%%%%%%%%%%%%%%%%%%%
%%%%%%%%%%%%%%%%%%%%%%%%%%%%%%%%%%%%%%%%%%%%%%%%%%%%%%%%%%%%%%%%%%%%%%%%%%%%%%%%%%%%%%%%%%%%%%%%%%%%%%%%%%%%%%%%%%%%%%%%%%%%%%%%%%%%%%%%%%%%%%%%%
\begin{tcbenvironment}|raw\_data|
%%%%%%%%%%%%%%%%%%%%%%%%%%%%%%%%%%%%%%%%%%%%%%%%%%%%%%%%%%%%%%%%%%%%%%%%%%%%%%%%%%%%%%%%%%%%%%%%%%%%%%%%%%%%%%%%%%%%%%%%%%%%%%%%%%%%%%%%%%%%%%%%%
\begin{tcbparbox}|\dirsep ion\dirsep pCT\_data\dirsep raw\_data|%
\bfdash directory where all raw experimental data files from a particular scan are stored in separate directories according to the scan date prior to creation of soft symbolic links named \docentry![fileset]{projection\_xxx.bin} and organized according to the naming/organizational scheme.
\end{tcbparbox}
%%%%%%%%%%%%%%%%%%%%%%%%%%%%%%%%%%%%%%%%%%%%%%%%%%%%%%%%%%%%%%%%%%%%%%%%%%%%%%%%%%%%%%%%%%%%%%%%%%%%%%%%%%%%%%%%%%%%%%%%%%%%%%%%%%%%%%%%%%%%%%%%%
\begin{tcbparbox}{tcbenumeratedStyle}
\begin{ThinEnum}[labelindent=1pt, leftmargin=*]
    \item \docentry[dir]{YY-MM-DD} : Folder containing all raw experimental data acquired from the scan beginning on \docentry![subcategory]{YY-MM-DD}
    \begin{ThinEnum}[labelindent=1pt, leftmargin=*]
        \item \docentry[fileset]{$<$Phantom$>$\_XXXX[\_AAA]\_xxx.dat} : raw experimental data for the object named \docentry![subcategory]{$<$Phantom$>$}, from run \# \docentry![subcategory]{XXXX[\_AAA]}, where \docentry![subcategory]{XXXX} is a 4 digit \# with leading zeros, \docentry![subcategory]{\_AAA} are optional ``\emph{subcategory tag(s)}'' indicating, e.g., a continuous scan (\docentry![subcategory]{\_Cont}), phantom position/section (inferior (\docentry![subcategory]{\_Inf}) or superior (\docentry![subcategory]{\_Sup}), top (\docentry![subcategory]{\_Top}) or bottom (\docentry![subcategory]{\_Bot}), etc.), and \docentry![subcategory]{xxx} is the gantry angle at which the data was acquired.
    \end{ThinEnum}
\end{ThinEnum}
\end{tcbparbox}
\end{tcbenvironment}
\endinput
%\clearpage 
%\clearpage%
\Subsection*(Preprocessed Data Hierarchy){\texttt{preprocessed\_data} Hierarchy}
%%%%%%%%%%%%%%%%%%%%%%%%%%%%%%%%%%%%%%%%%%%%%%%%%%%%%%%%%%%%%%%%%%%%%%%%%%%%%%%%%%%%%%%%%%%%%%%%%%%%%%%%%%%%%%%%%%%%%%%%%%%%%%%%%%%%%%%%%%%%%%%%%
%%%%%%%%%%%%%%%%%%%%%%%%%%%%%%%%%%%%%%%%%%%%%%%%%%%%%%%%%%%%%%%%%%%%%%%%%%%%%%%%%%%%%%%%%%%%%%%%%%%%%%%%%%%%%%%%%%%%%%%%%%%%%%%%%%%%%%%%%%%%%%%%%
%%%%%%%%%%%%%%%%%%%%%%%%%%%%%%%%%%%%%%%%%%%%%%%%%%%%%%%%%%%%%%%%%%%%%%%%%%%%%%%%%%%%%%%%%%%%%%%%%%%%%%%%%%%%%%%%%%%%%%%%%%%%%%%%%%%%%%%%%%%%%%%%%
\begin{tcbenvironment}|preprocessed\_data|
%%%%%%%%%%%%%%%%%%%%%%%%%%%%%%%%%%%%%%%%%%%%%%%%%%%%%%%%%%%%%%%%%%%%%%%%%%%%%%%%%%%%%%%%%%%%%%%%%%%%%%%%%%%%%%%%%%%%%%%%%%%%%%%%%%%%%%%%%%%%%%%%%
\begin{tcbparbox}|\dirsep ion\dirsep pCT\_data\dirsep preprocessed\_data|%
\bfdash directory containing the preprocessed experimental data organized by scan and processed dates
\end{tcbparbox}
%%%%%%%%%%%%%%%%%%%%%%%%%%%%%%%%%%%%%%%%%%%%%%%%%%%%%%%%%%%%%%%%%%%%%%%%%%%%%%%%%%%%%%%%%%%%%%%%%%%%%%%%%%%%%%%%%%%%%%%%%%%%%%%%%%%%%%%%%%%%%%%%%
%\begin{tcbparbox}{tcbenumeratedStyle}
\begin{tcbenumbox}
    \begin{ThinEnum}[labelindent=1pt, leftmargin=*]
        \item \docentry[dir]{YY-MM-DD} : Folder containing all processed experimental data corresponding to the raw experimental data acquired on \docentry![subcategory]{YY-MM-DD}
        \begin{ThinEnum}[labelindent=1pt, leftmargin=*]
            \item \docentry[dir]{YY-MM-DD} : Folder containing all processed experimental data generated on \docentry![subcategory]{YY-MM-DD} from the raw data
            \begin{ThinEnum}[labelindent=1pt, leftmargin=*]
                \item \docentry[file]{TVcorr.txt} : contains TV corrected WEPL calibration curve coefficients.
            \item \docentry[file]{WcalibTemp.txt} : temporary file containing WEPL calibration curve coefficients.
            \item \docentry[file]{Wcalib.txt} : contains final WEPL calibration curve coefficients.
            \item \docentry[fileset]{$<$Phantom$>$\_XXXX[\_AAA]\_xxx.dat.root.reco.root.bin} : preprocessed experimental data with tracker coordinates, recovery of missing hits when possible, and calibrated WEPL measurements for the object named \docentry![subcategory]{$<$Phantom$>$}, from run \# \docentry![subcategory]{XXXX[\_AAA]}, where \docentry![subcategory]{XXXX} is a 4 digit \# with leading zeros, \docentry![subcategory]{\_AAA} are optional ``\emph{subcategory tag(s)}'' indicating, e.g., a continuous scan (\docentry![subcategory]{\_Cont}), phantom position/section (inferior (\docentry![subcategory]{\_Inf}) or superior (\docentry![subcategory]{\_Sup}), top (\docentry![subcategory]{\_Top}) or bottom (\docentry![subcategory]{\_Bot}), etc.), and \docentry![subcategory]{xxx} is the gantry angle at which the data was acquired.
            \end{ThinEnum}
        \end{ThinEnum}
    \end{ThinEnum}
\end{tcbenumbox}
%\end{tcbparbox}
\end{tcbenvironment}
\endinput

%-------------------------------------------------------------------------------------------------------------------------------%
%\clearpage%
\Subsection(Unorganized User Data Hierarchy){\texttt{user\_data} (Unorganized Data) Hierarchy}
%-------------------------------------------------------------------------------------------------------------------------------%
%-------------------------------------------------------------------------------------------------------------------------------%
%-------------------------------------------------------------------------------------------------------------------------------%
\begin{tcbenvironment}|user\_data|
%%%%%%%%%%%%%%%%%%%%%%%%%%%%%%%%%%%%%%%%%%%%%%%%%%%%%%%%%%%%%%%%%%%%%%%%%%%%%%%%%%%%%%%%%%%%%%%%%%%%%%%%%%%%%%%%%%%%%%%%%%%%%%%%%%%%%%%%%%%%%%%%%
\begin{tcbparbox}|\dirsep ion\dirsep pCT\_data\dirsep user\_data|%
\bfdash directory containing unorganized input and output reconstruction data, allowing users to use and keep their
unorganized data separate from other data and maintain it in their preferred organizational scheme without it
interfering with the properly organized data.
\end{tcbparbox}
%%%%%%%%%%%%%%%%%%%%%%%%%%%%%%%%%%%%%%%%%%%%%%%%%%%%%%%%%%%%%%%%%%%%%%%%%%%%%%%%%%%%%%%%%%%%%%%%%%%%%%%%%%%%%%%%%%%%%%%%%%%%%%%%%%%%%%%%%%%%%%%%%
\begin{tcbparbox}{tcbenumeratedStyle}
\begin{ThinEnum}[labelindent=1pt, leftmargin=*]
    \item \docentry[constdir]{user\_data} : directory unique to Tardis compute nodes containing subdirectories for each
        user where they can transfer unorganized data they want to reconstruct and write the corresponding output
        reconstruction data/images
    \begin{ThinEnum}[labelindent=1pt, leftmargin=*]
        \item \docentry[dir]{\caratenclosed*{username}} : subdirectories for each pCT user for the unorganized input
            and output reconstruction data
    \end{ThinEnum}
\end{ThinEnum}
\end{tcbparbox}
\end{tcbenvironment}
\endinput
%\clearpage 
%-------------------------------------------------------------------------------------------------------------------------------%
\Subsection(Reconstruction Data Hierarchy){\texttt{reconstruction\_data} Hierarchy}
%-------------------------------------------------------------------------------------------------------------------------------%
%-------------------------------------------------------------------------------------------------------------------------------%
%-------------------------------------------------------------------------------------------------------------------------------%
\begin{tcbenvironment}|reconstruction\_data|
%%%%%%%%%%%%%%%%%%%%%%%%%%%%%%%%%%%%%%%%%%%%%%%%%%%%%%%%%%%%%%%%%%%%%%%%%%%%%%%%%%%%%%%%%%%%%%%%%%%%%%%%%%%%%%%%%%%%%%%%%%%%%%%%%%%%%%%%%%%%%%%%%
\begin{tcbparbox}|\dirsep ion\dirsep pCT\_data\dirsep reconstruction\_data|%
\bfdash directory containing the default data/images generated during reconstruction.  Additional data files and images can optionally be written to disk as well and a full list of these is given in the \autohyperlink[type=Section]{Reconstruction File List} section.
\end{tcbparbox}
    %%%%%%%%%%%%%%%%%%%%%%%%%%%%%%%%%%%%%%%%%%%%%%%%%%%%%%%%%%%%%%%%%%%%%%%%%%%%%%%%%%%%%%%%%%%%%%%%%%%%%%%%%%%%%%%%%%%%%%%%%%%%%%%%%%%%%%%%%%%%%%%%%
\begin{tcbparbox}{tcbenumeratedStyle}
    \begin{ThinEnum}
      \item \docentry[file]{execution\_log.csv} : global execution log containing entries with scan/object information
          and the settings/parameters used in reconstruction for each reconstructions performed to date, with new row
          entries added each time the reconstruction program is executed.
      \item \docentry[dir]{Phantom} : directory containing all of the experimental/simulated data and reconstructed images associated with this phantom/object.
        \begin{ThinEnum}
            \item \docentry[constdir]{Reference\_Images} : directory containing reference images (xCT, RSP, etc) relevant to analysis/comparison of the data/images for this object and data type.
            \item \docentry[constdir]{Experimental} : directory containing data and images generated from an experimental scan of the object.
            \begin{ThinEnum}
                \item \docentry[dir]{YY-MM-DD} : directory containing data and reconstructed images corresponding to the experimental scan of the object performed on this date.
                \begin{ThinEnum}
                    \item \docentry[dir]{XXXX[\_AAA]} : directory containing data/images corresponding to the 4-digit run \# \docentry![subcategory]{XXXX}, potentially including \emph{``subcategory tag(s)''} of the form \docentry![subcategory]{\_AAA} indicating, e.g., a continuous scan (\docentry![subcategory]{\_Cont}), phantom position/section (inferior (\docentry![subcategory]{\_Inf}) or superior (\docentry![subcategory]{\_Sup}), top (\docentry![subcategory]{\_Top}) or bottom (\docentry![subcategory]{\_Bot}), etc.).
                    \begin{ThinEnum}
                        \item \docentry[constdir]{Input} : directory containing raw data generated by scan of object from each gantry angle and transmitted by event builder.
                        \begin{ThinEnum}
                            \item \docentry[fileset]{raw\_xxx.bin} : binary files containing trigger/tracker/energy detector data from event builder associated with gantry angle \docentry![subcategory]{xxx} =\{\docentry![subcategory]{001}, \docentry![subcategory]{002}, \docentry![subcategory]{003}, $\cdots\}$.
                        \end{ThinEnum}
                        \item \docentry[constdir]{Output} : directory containing calibration and post processed data generated from analysis of raw data and used as input to image reconstruction.
                        \begin{ThinEnum}
                            \item \docentry[dir]{YY-MM-DD} : directory containing the post processed \docentry![fileset]{projection\_xxx.bin} data generated on this date and the reconstructions using this data.
                            \begin{ThinEnum}
                                \item \docentry[file]{readme.txt} : contains input raw data info, phantom name, and run date.
                                \item \docentry[file]{TVcorr.txt} : contains TV corrected WEPL calibration curve coefficients.
                        \item \docentry[file]{WcalibTemp.txt} : temporary file containing WEPL calibration curve coefficients.
                                    \item \docentry[file]{Wcalib.txt} : contains final WEPL calibration curve coefficients.
                                    \item \docentry[fileset]{projection\_xxx.bin} : preprocessed data files containing tracker coordinates and WEPL data for gantry angle $\text{\docentry![subcategory]{xxx}} =\{\text{\docentry![subcategory]{001}}, \text{\docentry![subcategory]{002}}, \text{\docentry![subcategory]{003}}, \cdots\}$ used as input to image reconstruction.
                        \item \docentry[constdir]{Reconstruction} : directory containing preprocessed data and reconstructed images generated using the \docentry![fileset]{projection\_xxx.bin} data along with reference images relevant to the object.
                                    \begin{ThinEnum}
                                        \item \docentry[file]{settings.cfg} : configuration file containing key/value pairs specifying scan/phantom properties (phantom, run date/\#/tag(s), etc.) and default reconstruction settings/parameters.
                                    \item \docentry[dir]{YY-MM-DD} : directory containing the preprocessed data generated on this date and the reconstructed images generated from this data.
                                    \begin{ThinEnum}
                                            \item \docentry[file]{execution\_log.txt} : execution times for various portions of preprocessing and/or reconstruction and total program execution time.
                                            \item \docentry[file]{FBP.txt} : text image of filtered back projection (FBP) image.
                                            \item \docentry[file]{FBP.png} : conversion of \docentry![file]{FBP.txt} to PNG image.
                                            \item \docentry[file]{FBP\_med\_filtered.txt} : text image result of applying median filter to the filtered back projection (FBP) image.
                                            \item \docentry[file]{FBP\_med\_filtered.png} : conversion of \docentry![file]{FBP\_avg\_filtered.txt} to PNG image.
                                            \item \docentry[file]{hull.txt} : text image of selected object hull in 1s/0s.
                                            \item \docentry[file]{hull.png} : conversion of \docentry![file]{hull.txt} to PNG image.
                                            \item \docentry[file]{hull\_avg\_filtered.txt} : text image result of applying average filter to the hull image.
                            \item \docentry[file]{hull\_avg\_filtered.png} : conversion of \docentry![file]{hull\_avg\_filtered.txt} to PNG image.
                            \item \docentry[file]{settings\_log.cfg} : copy of \docentry![file]{settings.cfg} with any changes made to parameters/options applied at execution, if any.
                                            \item \docentry[file]{TV\_measurements.txt} : total variation (TV) measurements before/after each iteration
                                            \item \docentry[file]{x\_0.txt} : text image of initial iterate.
                                        \item \docentry[file]{x\_0.png} : conversion of \docentry![file]{x\_0.txt} to PNG image.
                                        \item \docentry[constdir]{Images} : directory containing reconstructed images generated using this preprocessed data.
                                        \begin{ThinEnum}
                                            \item \docentry[dir]{YY-MM-DD} : directory containing the reconstructed images generated on this date using the preprocessed data above.
                                                \begin{ThinEnum}
                                                    \item \docentry[fileset]{x\_k.txt} : text image of reconstructed image $\boldsymbol{x^k}$ after $\boldsymbol{k}$ iterations.
                                                    \item \docentry[fileset]{x\_k.png} : PNG image of reconstructed image $\boldsymbol{x^k}$ after $\boldsymbol{k}$ iterations.
                                                \end{ThinEnum}
                                        \end{ThinEnum}
                                    \end{ThinEnum}
                                \end{ThinEnum}
                            \end{ThinEnum}
                        \end{ThinEnum}
                    \end{ThinEnum}
                \end{ThinEnum}
            \end{ThinEnum}
            %\newpage
            \item \docentry[constdir]{Simulated} : directory containing data and images generated from simulated scans of the object.
            \begin{ThinEnum}
                \item \docentry[dir]{G\_YY-MM-DD} : directory containing data and reconstructed images from the GEANT4 simulated scan of the object generated on this date..
                \begin{ThinEnum}
                    \item \docentry[dir]{XXXX[\_AAA]} : directory containing data/images corresponding to the 4-digit run \# \docentry![subcategory]{XXXX}, potentially including \emph{``subcategory tag(s)''} of the form \docentry![subcategory]{\_AAA} indicating, e.g., a continuous scan (\docentry![subcategory]{\_Cont}), phantom position/section (inferior (\docentry![subcategory]{\_Inf}) or superior (\docentry![subcategory]{\_Sup}), top (\docentry![subcategory]{\_Top}) or bottom (\docentry![subcategory]{\_Bot}), etc.).
                    \begin{ThinEnum}
                        \item \docentry[constdir]{Input} : directory containing raw data files generated by simulated scan of object for each gantry angle.
                        \begin{ThinEnum}
                            \item \docentry[fileset]{raw\_xxx.bin} : binary files containing trigger/tracker/energy detector data from event builder associated with gantry angle \docentry![subcategory]{xxx} =\{\docentry![subcategory]{001}, \docentry![subcategory]{002}, \docentry![subcategory]{003}, $\cdots\}$.
                        \end{ThinEnum}
                        \item \docentry[constdir]{Output} : directory containing calibration and post processed data generated from analysis of raw data and used as input to image reconstruction.
                        \begin{ThinEnum}
                            \item \docentry[dir]{YY-MM-DD} : directory containing the post processed \docentry![fileset]{projection\_xxx.bin} data generated on this date and the reconstructions using this data.
                            \begin{ThinEnum}
                                \item \docentry[file]{readme.txt} : contains input raw data info, phantom name, and run date.
                                \item \docentry[file]{TVcorr.txt} : contains TV corrected WEPL calibration curve coefficients.
                        \item \docentry[file]{WcalibTemp.txt} : temporary file containing WEPL calibration curve coefficients.
                                    \item \docentry[file]{Wcalib.txt} : contains final WEPL calibration curve coefficients.
                                \item \docentry[fileset]{projection\_xxx.bin} : preprocessed data files containing tracker coordinates and WEPL data for gantry angle $\text{\docentry![subcategory]{xxx}} =\{\text{\docentry![subcategory]{001}}, \text{\docentry![subcategory]{002}}, \text{\docentry![subcategory]{003}}, \cdots\}$ used as input to image reconstruction.
                                \item \docentry[constdir]{Reconstruction} : directory containing preprocessed data and reconstructed images generated using the \docentry![fileset]{projection\_xxx.bin} data along with reference images relevant to the object.
                                    \begin{ThinEnum}
                                        \item \docentry[file]{settings.cfg} : configuration file containing key/value pairs specifying scan/phantom properties (phantom, run date/\#/tag(s), etc.) and default reconstruction settings/parameters.
                                    \item \docentry[dir]{YY-MM-DD} : directory containing the preprocessed data generated on this date and the reconstructed images generated from this data.
                                \begin{ThinEnum}
                                            \item \docentry[file]{execution\_log.txt} : execution times for various portions of preprocessing and/or reconstruction and total program execution time.
                                            \item \docentry[file]{FBP.txt} : text image of filtered back projection (FBP) image.
                                            \item \docentry[file]{FBP.png} : conversion of \docentry![file]{FBP.txt} to PNG image.
                                            \item \docentry[file]{FBP\_med\_filtered.txt} : text image result of applying median filter to the filtered back projection (FBP) image.
                                            \item \docentry[file]{FBP\_med\_filtered.png} : conversion of \docentry![file]{FBP\_avg\_filtered.txt} to PNG image.
                                            \item \docentry[file]{hull.txt} : text image of selected object hull in 1s/0s.
                                            \item \docentry[file]{hull.png} : conversion of \docentry![file]{hull.txt} to PNG image.
                                            \item \docentry[file]{hull\_avg\_filtered.txt} : text image result of applying average filter to the hull image.
                            \item \docentry[file]{hull\_avg\_filtered.png} : conversion of \docentry![file]{hull\_avg\_filtered.txt} to PNG image.
                            \item \docentry[file]{settings\_log.cfg} : copy of \docentry![file]{settings.cfg} with any changes made to parameters/options applied at execution, if any.
                                            \item \docentry[file]{TV\_measurements.txt} : total variation (TV) measurements before/after each iteration
                                            \item \docentry[file]{x\_0.txt} : text image of initial iterate.
                                        \item \docentry[file]{x\_0.png} : conversion of \docentry![file]{x\_0.txt} to PNG image.
                                        \item \docentry[constdir]{Images} : directory containing reconstructed images generated using this preprocessed data.
                                        \begin{ThinEnum}
                                            \item \docentry[dir]{YY-MM-DD} : directory containing the reconstructed images generated on this date using the preprocessed data above.
                                                \begin{ThinEnum}
                                                    \item \docentry[fileset]{x\_k.txt} : text image of reconstructed image $\boldsymbol{x^k}$ after $\boldsymbol{k}$ iterations.
                                                    \item \docentry[fileset]{x\_k.png} : PNG image of reconstructed image $\boldsymbol{x^k}$ after $\boldsymbol{k}$ iterations.
                                                \end{ThinEnum}
                                        \end{ThinEnum}
                                    \end{ThinEnum}
                                \end{ThinEnum}
                            \end{ThinEnum}
                        \end{ThinEnum}
                    \end{ThinEnum}
                \end{ThinEnum}
                \item \docentry[dir]{T\_YY-MM-DD} : directory containing data and reconstructed images corresponding to all TOPAS simulated scans of the object generated on this date.
                \begin{ThinEnum}
                    \item \docentry[dir]{XXXX[\_AAA]} : directory containing data/images corresponding to the 4-digit run \# \docentry![subcategory]{XXXX}, potentially including \emph{``subcategory tag(s)''} of the form \docentry![subcategory]{\_AAA} indicating, e.g., a continuous scan (\docentry![subcategory]{\_Cont}), phantom position/section (inferior (\docentry![subcategory]{\_Inf}) or superior (\docentry![subcategory]{\_Sup}), top (\docentry![subcategory]{\_Top}) or bottom (\docentry![subcategory]{\_Bot}), etc.).
                    \begin{ThinEnum}
                        \item \docentry[constdir]{Input} : directory containing raw data files generated by simulated scan of object for each gantry angle.
                        \begin{ThinEnum}
                            \item \docentry[fileset]{raw\_xxx.bin} : binary files containing trigger/tracker/energy detector data from event builder associated with gantry angle \docentry![subcategory]{xxx} =\{\docentry![subcategory]{001}, \docentry![subcategory]{002}, \docentry![subcategory]{003}, $\cdots\}$.
                        \end{ThinEnum}
                        \item \docentry[constdir]{Output} : directory containing calibration and post processed data generated from analysis of raw data and used as input to image reconstruction.
                        \begin{ThinEnum}
                            \item \docentry[dir]{YY-MM-DD} : directory containing the post processed \docentry![fileset]{projection\_xxx.bin} data generated on this date and the reconstructions using this data.
                            \begin{ThinEnum}
                                \item \docentry[file]{readme.txt} : contains input raw data info, phantom name, and run date.
                                \item \docentry[file]{TVcorr.txt} : contains TV corrected WEPL calibration curve coefficients.
                        \item \docentry[file]{WcalibTemp.txt} : temporary file containing WEPL calibration curve coefficients.
                                    \item \docentry[file]{Wcalib.txt} : contains final WEPL calibration curve coefficients.
                                \item \docentry[fileset]{projection\_xxx.bin} : preprocessed data files containing tracker coordinates and WEPL data for gantry angle $\text{\docentry![subcategory]{xxx}} =\{\text{\docentry![subcategory]{001}}, \text{\docentry![subcategory]{002}}, \text{\docentry![subcategory]{003}}, \cdots\}$ used as input to image reconstruction.
                                \item \docentry[constdir]{Reconstruction} : directory containing preprocessed data and reconstructed images generated using the \docentry![fileset]{projection\_xxx.bin} data along with reference images relevant to the object.
                                    \begin{ThinEnum}
                                        \item \docentry[file]{settings.cfg} : configuration file containing key/value pairs specifying scan/phantom properties (phantom, run date/\#/tag(s), etc.) and default reconstruction settings/parameters.
                                    \item \docentry[dir]{YY-MM-DD} : directory containing the preprocessed data generated on this date and the reconstructed images generated from this data.
                                    \begin{ThinEnum}
                                            \item \docentry[file]{execution\_log.txt} : execution times for various portions of preprocessing and/or reconstruction and total program execution time.
                                            \item \docentry[file]{FBP.txt} : text image of filtered back projection (FBP) image.
                                            \item \docentry[file]{FBP.png} : conversion of \docentry![file]{FBP.txt} to PNG image.
                                            \item \docentry[file]{FBP\_med\_filtered.txt} : text image result of applying median filter to the filtered back projection (FBP) image.
                                            \item \docentry[file]{FBP\_med\_filtered.png} : conversion of \docentry![file]{FBP\_avg\_filtered.txt} to PNG image.
                                            \item \docentry[file]{hull.txt} : text image of selected object hull in 1s/0s.
                                            \item \docentry[file]{hull.png} : conversion of \docentry![file]{hull.txt} to PNG image.
                                            \item \docentry[file]{hull\_avg\_filtered.txt} : text image result of applying average filter to the hull image.
                            \item \docentry[file]{hull\_avg\_filtered.png} : conversion of \docentry![file]{hull\_avg\_filtered.txt} to PNG image.
                            \item \docentry[file]{settings\_log.cfg} : copy of \docentry![file]{settings.cfg} with any changes made to parameters/options applied at execution, if any.
                                            \item \docentry[file]{TV\_measurements.txt} : total variation (TV) measurements before/after each iteration
                                            \item \docentry[file]{x\_0.txt} : text image of initial iterate.
                                        \item \docentry[file]{x\_0.png} : conversion of \docentry![file]{x\_0.txt} to PNG image.
                                        \item \docentry[constdir]{Images} : directory containing reconstructed images generated using this preprocessed data.
                                        \begin{ThinEnum}
                                            \item \docentry[dir]{YY-MM-DD} : directory containing the reconstructed images generated on this date using the preprocessed data above.
                                                \begin{ThinEnum}
                                                    \item \docentry[fileset]{x\_k.txt} : text image of reconstructed image $\boldsymbol{x^k}$ after $\boldsymbol{k}$ iterations.
                                                    \item \docentry[fileset]{x\_k.png} : PNG image of reconstructed image $\boldsymbol{x^k}$ after $\boldsymbol{k}$ iterations.%\\
                                                \end{ThinEnum}
                                        \end{ThinEnum}
                                    \end{ThinEnum}
                                \end{ThinEnum}
                            \end{ThinEnum}
                        \end{ThinEnum}
                    \end{ThinEnum}
                \end{ThinEnum}
            \end{ThinEnum}
        \end{ThinEnum}
    \end{ThinEnum}
%%%%%%%%%%%%%%%%%%%%%%%%%%%%%%%%%%%%%%%%%%%%%%%%%%%%%%%%%%%%%%%%%%%%%%%%%%%%%%%%%%%%%%%%%%%%%%%%%%%%%%%%%%%%%%%%%%%%%%%%%%%%%%%%%%%%%%%%%%%%%%%%%
\end{tcbparbox}
%%%%%%%%%%%%%%%%%%%%%%%%%%%%%%%%%%%%%%%%%%%%%%%%%%%%%%%%%%%%%%%%%%%%%%%%%%%%%%%%%%%%%%%%%%%%%%%%%%%%%%%%%%%%%%%%%%%%%%%%%%%%%%%%%%%%%%%%%%%%%%%%%
\end{tcbenvironment}
%%%%%%%%%%%%%%%%%%%%%%%%%%%%%%%%%%%%%%%%%%%%%%%%%%%%%%%%%%%%%%%%%%%%%%%%%%%%%%%%%%%%%%%%%%%%%%%%%%%%%%%%%%%%%%%%%%%%%%%%%%%%%%%%%%%%%%%%%%%%%%%%%
\endinput
%%%%%%%%%%%%%%%%%%%%%%%%%%%%%%%%%%%%%%%%%%%%%%%%%%%%%%%%%%%%%%%%%%%%%%%%%%%%%%%%%%%%%%%%%%%%%%%%%%%%%%%%%%%%%%%%%%%%%%%%%%%%%%%%%%%%%%%%%%%%%%%%%
%%%%%%%%%%%%%%%%%%%%%%%%%%%%%%%%%%%%%%%%%%%%%%%%%%%%%%%%%%%%%%%%%%%%%%%%%%%%%%%%%%%%%%%%%%%%%%%%%%%%%%%%%%%%%%%%%%%%%%%%%%%%%%%%%%%%%%%%%%%%%%%%%
%%%%%%%%%%%%%%%%%%%%%%%%%%%%%%%%%%%%%%%%%%%%%%%%%%%%%%%%%%%%%%%%%%%%%%%%%%%%%%%%%%%%%%%%%%%%%%%%%%%%%%%%%%%%%%%%%%%%%%%%%%%%%%%%%%%%%%%%%%%%%%%%%
\item \docentry[file]{bin\_counts.txt} : linearized bin \# for each proton history, where linearized bin \# = t\_bin + angle\_bin * T\_BINS + v\_bin * T\_BINS * ANGULAR\_BINS.
\item {\docentry[file]{mean\_rel\_ut\_angle.txt} : mean relative ut angle ($\angle ut_{out} -\angle ut_{in}$) by linearized bin \#.}
\item \docentry[file]{mean\_rel\_uv\_angle.txt} : mean relative uv angle ($\angle uv_{out} -\angle uv_{in}$) by linearized bin \#.
\item \docentry[file]{mean\_WEPL.txt} : mean WEPL value by linearized bin \#.
\item \docentry[file]{stddev\_rel\_ut\_angle.txt} : standard deviation of the relative ut angle ($\angle ut_{out} -\angle ut_{in}$) by linearized bin \#.
\item \docentry[file]{stddev\_rel\_uv\_angle.txt} : standard deviation of the relative uv angle ($\angle uv_{out} -\angle uv_{in}$) by linearized bin \#.
\item \docentry[file]{stddev\_WEPL.txt} : standard deviation of the WEPL value by linearized bin \#.
\item \docentry[file]{sinogram.txt} : mean WEPL after statistical cuts with the $t_{bin}$ and angular bin $\theta_{bin}$ plane for each vertical bin $v_{bin}$ stacked on each other.
\item \docentry[file]{sin\_table.bin} : file containing the tabulated values of sine function
\item \docentry[file]{cos\_table.bin} : file containing the tabulated values of cosine function
\item \docentry[file]{coefficient.bin} : file containing the tabulated scattering coefficient values for $\Sigma_1$/$\Sigma_2$ for $u_2-u_1$/$u_1$ values
\item \docentry[file]{poly\_1\_2.bin} : file containing the tabulated MLP polynomial values with coefficients $\{1,2,3,4,5,6\}$
\item \docentry[file]{poly\_2\_3.bin} : file containing the tabulated MLP polynomial values with coefficients $\{2,3,4,5,6,7\}$
\item \docentry[file]{poly\_3\_4.bin} : file containing the tabulated MLP polynomial values with coefficients $\{3,4,5,6,7,8\}$
\item \docentry[file]{poly\_2\_6.bin} : file containing the tabulated MLP polynomial values with coefficients $\{2,6,12,20,30,42\}$
\item \docentry[file]{poly\_3\_12.bin} : file containing the tabulated MLP polynomial values with coefficients $\{3,12,30,60,105,168\}$
\item \docentry[file]{MLP.bin} : binary file with MLP path data for each history entering hull.
\item \docentry[file]{WEPL.bin} : binary file specifying WEPL value for each history entering hull.
\item \docentry[file]{histories.bin} : binary file specifying entry/exit coordinates/angles, bin number, gantry angle, and hull entry x/y/z voxel \# for each history entering hull.

\gls{naiive}\\
\gls{computer}\\
%-------------------------------------------------------------------------------------------------------------------------------%
%-------------------------------------------------------------------------------------------------------------------------------%
%-------------------------------------------------------------------------------------------------------------------------------%
\Chapter{Code Hierarchy Diagrams}
First line after chapter
\index{hello}
\idxTestContent
%-------------------------------------------------------------------------------------------------------------------------------%
\Section(pCT code Hierarchy Diagram){\texttt{pCT\_code} Hierarchy Diagram}%
[%
	the diagram below shows the hierarchy of \docentry[constdir]{pCT\_code} directories listed and described in the \autohyperlink[type=Subsection]{pCT Code Hierarchy} section, including the clones of the most commonly used GitHub accounts/repositories relevant to pCT (described in the \autohyperlink[type=Section]{GitHub Accounts/Repositories} section).%
]%
%-------------------------------------------------------------------------------------------------------------------------------%
%-------------------------------------------------------------------------------------------------------------------------------%
%-------------------------------------------------------------------------------------------------------------------------------%
%\centering%
\begin{tcbdiagram}
%\centering%
\matrix[matrix of nodes, row sep=4mm,column sep=1.5mm]
{%
&\node(parentdot)[pt]{};\\
&\node(01)[D]{ion};\\
&\node(11)[D]{pCT\_code};\\
\node(20)[pt]{};&\node(21)[pt]{};&&\node(23)[pt]{};\\
\node(30)[D]{user\_code};    &&&\node(33)[D]{git};\\[0mm]
\node(40)[D]{$<$username$>$};&\node(41)[pt]{};&\node(42)[pt]{};&\node(43)[pt]{};&&\node(45)[pt]{};\\
&\node(51)[D]{BlakeSchultze};&\node(52)[D]{BaylorICTHUS};&&&\node(55)[D]{pCT-collaboration};\\[1mm]
\node(60)[pt]{};&\node(61)[pt]{};&\node(62)[F]{pCT\_Reconstruction};&\node(63)[pt]{};&\node(64)[pt]{};&\node(65)[pt]{};&\node(66)[pt]{};&\node(67)[pt]{};\\[1mm]
\node(70)[F]{pCT\_Reconstruction};&\node(71)[F]{pCT\_Documentation};&&\node(73)[F]{pCT-docs};&\node(74)[F]{pCT\_Tools};&\node(75)[F]{Reconstruction\_BU};&\node(76)[F]{pct-recon-copy};&\node(77)[F]{Preprocessing};\\[1mm]
};
\graph[use existing nodes, edge label, edges=thick]
{
(parentdot)->(01)->(11)->(21)--(20)->(30)->(40);
(21)--(23)->(33)--(43)--(42)--(41)->(51)--(61)--(60)->(70);
(43)--(45)->(55)--(65)--(63)--(67)->(77);
(42)->(52)->(62);
(61)->(71);
(63)->(73);
(64)->(74);
(65)->(75);
(66)->(76);
};
\end{tcbdiagram}
%*******************************************************************************************************************************%
%*******************************************************************************************************************************%
%*******************************************************************************************************************************%
\endinput 
%-------------------------------------------------------------------------------------------------------------------------------%
%\clearpage%
\Section*(Organized Data Hierarchy Diagram){\texttt{organized\_data} Hierarchy Diagram}%
[%
	\vspace{-2mm}%
	the diagram below shows the hierarchy of \autohyperlink[type=Subsection,label={\docentry[constdir]{organized\_data}}]{Organized Data Hierarchy} directories.  The vertical ellipses (\,\protect\tinyvdots\,) indicate that the subdirectories below this level of the branch are identical to those below this same level of the branch that shows these subdirectories explicitly.%
]
%-------------------------------------------------------------------------------------------------------------------------------%
%-------------------------------------------------------------------------------------------------------------------------------%
%-------------------------------------------------------------------------------------------------------------------------------%
\begin{tcbdiagram}
%\centering%
\matrix[matrix of nodes, row sep=4mm,column sep=1.5mm]
{%
    &&&\node(parentdot)[pt]{};&&&&\\
    &&&\node(parentdir)[D]{ion};&&&&\\
    &&&\node(pctdata)[D]{pCT\_data};&&&&\\
    &&&\node(organizeddata)[D]{organized\_data};&&&&\\
    &\node(top1)[pt]{};&&\node(top3)[pt]{};&\node(top4)[pt]{};&&&\\
    &\node(001)[D]{Experimental};    &&&\node(004)[D]{Simulated};&\\[-2mm]
    &\node(01)[pt]{};&&\node(03)[pt]{};&\node(04)[pt]{};&\node(05)[pt]{};&&\\
    &\node(11)[D]{YY-MM-DD};        &&\node(13)[D]{G\_YY-MM-DD};&&\node(15)[D]{T\_YY-MM-DD};\\[1mm]
    &&&\node(0023)[textonly]{$\vdots$};&&\node(0025)[textonly]{$\vdots$};&&\\[-13.15mm]
    %&&&&&\\[-5mm]
    &\node(21)[D]{XXXX[\_AAA]};     &&\node(23)[textonly]{};&&\node(25)[textonly]{};\\[0mm]
    \node(30)[pt]{};&\node(31)[pt]{}; & \node(32)[pt]{};&&&\\
    \node(40)[D]{Input};       &&\node(42)[D]{Output};       &&&&\\
    \node(50)[F]{raw\_xxx.bin};&&\node(52)[D]{YY-MM-DD};     &&&&\\[-2mm]
%-------------------------------------------------------------------------------------------------------------------------------%
    &\node(61)[pt]{};          &\node(62)[pt]{};    &\node(63)[pt]{};&\node(64)[pt]{};&\node(65)[pt]{};\\
    &\node(71)[F]{Wcalib.txt}; &\node(72)[D]{Reconstruction}; &\node(73)[F]{TVcorr.txt};   &\node(74)[F]{WcalibTemp.txt};&\node(75)[F]{projection\_xxx.bin};&&&&&&&&\\[0mm]
    &                  &\node(82)[pt]{};&\node(83)[pt]{};&\\
    &                          &\node(92)[D]{YY-MM-DD};       &\node(93)[F]{settings.cfg};&&&&&&&&&&\\[1mm]
%-------------------------------------------------------------------------------------------------------------------------------%
    \node(100)[pt]{};&\node(101)[pt]{};&\node(102)[pt]{};    &\node(103)[pt]{};&\node(104)[pt]{};&\node(105)[pt]{};\\
    \node(110)[F]{hull.txt};&\node(111)[F]{FBP.txt};&\node(112)[D]{Images};&\node(113)[F]{x\_0.txt};&\node(114)[F]{TV\_measurements.txt};&\node(115)[F]{execution\_log.txt};
    &&&&&&&&\\
%-------------------------------------------------------------------------------------------------------------------------------%
    && \node(122)[D]{YY-MM-DD};\\[-2mm]
    &\node(131)[pt]{};&\node(132)[pt]{};&\node(133)[pt]{};&&\\
    &\node(141)[F]{x\_k.dcm};&\node(142)[F]{x\_k.txt};&\node(143)[F]{x\_k.png};&&&&&&&&&&\\
};
\graph[use existing nodes, edge label, edges=thick]
{
    (parentdot)->(parentdir)->(pctdata)->(organizeddata)--(top3)--(top1)->(001)--(01)->(11)->(21)--(31)--(30)->(40)->(50);
    (top3)--(top4)->(004)--(04)--(05)->(15)->(25);
    (04)--(03)->(13)->(23);
    (31)--(32)->(42)->(52)--(62)->(72)--(82)->(92)--(102)->(112)->(122)--(132)->(142);
    (61)--(62)--(63)--(64)--(65);
    (61)->(71);
    (62)->(72);
    (63)->(73);
    (64)->(74);
    (65)->(75);
    (82)--(83)->(93);
    (100)--(101)--(102)--(103)--(104)--(105);
    (100)->(110);
    (101)->(111);
    (102)->(112);
    (103)->(113);
    (104)->(114);
    (105)->(115);
    (131)--(132)--(133);
    (131)->(141);
    (132)->(142);
    (133)->(143)
};
\end{tcbdiagram}
%*******************************************************************************************************************************%
%*******************************************************************************************************************************%
%*******************************************************************************************************************************%
\endinput 
%-------------------------------------------------------------------------------------------------------------------------------%
%-------------------------------------------------------------------------------------------------------------------------------%
%-------------------------------------------------------------------------------------------------------------------------------%
\Chapter{GitHub Sources}
%\clearpage%
\Section{GitHub Accounts/Repositories}
%%%%%%%%%%%%%%%%%%%%%%%%%%%%%%%%%%%%%%%%%%%%%%%%%%%%%%%%%%%%%%%%%%%%%%%%%%%%%%%%%%%%%%%%%%%%%%%%%%%%%%%%%%%%%%%%%%%%%%%%%%%%%%%%%%%%%%%%%%%%%%%%%
%%%%%%%%%%%%%%%%%%%%%%%%%%%%%%%%%%%%%%%%%%%%%%%%%%%%%%%%%%%%%%%%%%%%%%%%%%%%%%%%%%%%%%%%%%%%%%%%%%%%%%%%%%%%%%%%%%%%%%%%%%%%%%%%%%%%%%%%%%%%%%%%%
%%%%%%%%%%%%%%%%%%%%%%%%%%%%%%%%%%%%%%%%%%%%%%%%%%%%%%%%%%%%%%%%%%%%%%%%%%%%%%%%%%%%%%%%%%%%%%%%%%%%%%%%%%%%%%%%%%%%%%%%%%%%%%%%%%%%%%%%%%%%%%%%%
\begin{tcbenvironment}|GitHub|
%%%%%%%%%%%%%%%%%%%%%%%%%%%%%%%%%%%%%%%%%%%%%%%%%%%%%%%%%%%%%%%%%%%%%%%%%%%%%%%%%%%%%%%%%%%%%%%%%%%%%%%%%%%%%%%%%%%%%%%%%%%%%%%%%%%%%%%%%%%%%%%%%
\begin{tcbparbox}|https:\dirsep\dirsep github.com\dirsep\caratenclosed*{GitHub account}\dirsep\caratenclosed*{GitHub repository}|
\bfdash below is a description of the GitHub accounts/repositories containing the tools/programs relevant to pCT and documentation for pCT software/hardware, code/data storage and management, collaborator projects and contact information, phantom naming/properties, and other useful pCT information.  These have been cloned to Kodiak/Tardis and organized according to the scheme described in \autohyperlink[type=Section]{Code Organization} and shown in the \autohyperlink[type=Section, label=pCT\_code Hierarchy Diagram]{pCT code Hierarchy Diagram}, thereby providing users with easy/immediate access to the source code in these repositories.
\end{tcbparbox}
%%%%%%%%%%%%%%%%%%%%%%%%%%%%%%%%%%%%%%%%%%%%%%%%%%%%%%%%%%%%%%%%%%%%%%%%%%%%%%%%%%%%%%%%%%%%%%%%%%%%%%%%%%%%%%%%%%%%%%%%%%%%%%%%%%%%%%%%%%%%%%%%%
\def\currentparsep{3.0pt}
\begin{tcbparbox}{tcbenumeratedStyle}
\begin{ThinEnum}[parsep=\currentparsep]
    \item \docentry[gitaccount]{pCT-collaboration} : contains repositories for pCT data acquisition, simulation, preprocessing, and reconstruction software and documentation describing the software/hardware, management of code/data, and other information relevant to pCT (e.g. collaborator list, hardware descriptions, phantom properties, etc.).
    \begin{ThinEnum}[parsep=\currentparsep]
        \item \docentry[gitrepo]{pCT\_Tools} : contains bash functions/scripts and other tools useful for navigating data/code and configuring/running programs on Kodiak and Tardis compute nodes (along with documentation describing them and their purpose/usage) including a default \docentry[script]{\texttt{.bash\_profile}} which sources \docentry[script]{\texttt{pct\_user\_script.sh}} to configure user sessions for the current host/node and \docentry[script]{\texttt{load\_pct\_functions.sh}} to automatically load the aforementioned bash functions useful during a user terminal session (see \docentry[file]{documentation.pdf} for
        \item \docentry[gitrepo]{pCT-docs} : contains documentation on the pCT data/code naming and organizational scheme, collaborator's project involvement and contact info, and phantom properties/manuals/naming (including relevant subcategory tags).  The importance and naming of additional documentation from the original \docentry[gitrepo]{BlakeSchultze\dirsep pCT\_Documentation} repository are currently being evaluated for migration.
        \item \docentry[gitrepo]{pypct} : Python helpers for proton CT
        \item \docentry[gitrepo]{pct-acquire} : Proton CT data acquisition software for the Phase 2 pCT scanner system
        \item \docentry[gitrepo]{pct-sim} : GEANT4 program for simulating scans with the Phase 2 pCT scanner system
        \item \docentry[gitrepo]{Preprocessing} : program for preprocessing raw data to calculate tracker plane coordinates from tracker chip/channel/strip values and generate calibrated WEPL values from energy detector measurements.
        \item \docentry[gitrepo]{pct-recon-copy} : original Penfold/Hurley pCT reconstruction program now with a \docentry[gitbranch]{Baylor} branch configuring the \docentry[script]{Makefile} for Tardis execution and input/output directory execution parameters required for batch script submission of reconstruction job(s) to the GPU execution queue.
        \item \docentry[gitrepo]{Reconstruction\_BU} : contains only the current and previous release versions of Baylor's reconstruction program as developed in \docentry[gitaccount]{BaylorICTHUS\dirsep pCT\_Reconstruction} (no code development is performed here).
    \end{ThinEnum}
%%%%%%%%%%%%%%%%%%%%%%%%%%%%%%%%%%%%%%%%%%%%%%%%%%%%%%%%%%%%%%%%%%%%%%%%%%%%%%%%%%%%%%%%%%%%%%%%%%%%%%%%%%%%%%%%%%%%%%%%%%%%%%%%%%%%%%%%%%%%%%%%%
%\newpage
%%%%%%%%%%%%%%%%%%%%%%%%%%%%%%%%%%%%%%%%%%%%%%%%%%%%%%%%%%%%%%%%%%%%%%%%%%%%%%%%%%%%%%%%%%%%%%%%%%%%%%%%%%%%%%%%%%%%%%%%%%%%%%%%%%%%%%%%%%%%%%%%%
%%%%%%%%%%%%%%%%%%%%%%%%%%%%%%%%%%%%%%%%%%%%%%%%%%%%%%%%%%%%%%%%%%%%%%%%%%%%%%%%%%%%%%%%%%%%%%%%%%%%%%%%%%%%%%%%%%%%%%%%%%%%%%%%%%%%%%%%%%%%%%%%%
    \item \docentry[gitaccount]{BaylorICTHUS} : Baylor's pCT programs, tools, and documentation.
    \begin{ThinEnum}[parsep=\currentparsep]
        \item \docentry[gitrepo]{pCT\_Reconstruction} : used in developing the release version of Baylor's pCT reconstruction program and containing branches for each of Baylor's pCT developers (Blake, Paniz, Sarah, ...) for independent development relevant to their work.  Developments made in a developer branch and proposed for integration in the next release version go through a review and testing process to verify the code and its impact on the full program.  Developments passing this verification process are then merged into the \docentry[gitbranch]{release\_development} branch.  When critical developments are merged into the \docentry[gitbranch]{release\_development} branch, this branch is then merged into the \docentry[gitbranch]{release} branch and the resulting code is then pushed to the \docentry[gitrepo]{pCT-collaboration\dirsep Reconstruction\_BU} repository, as this is the source for pCT users to acquire the current and previous release versions of the program.
    \end{ThinEnum}
%%%%%%%%%%%%%%%%%%%%%%%%%%%%%%%%%%%%%%%%%%%%%%%%%%%%%%%%%%%%%%%%%%%%%%%%%%%%%%%%%%%%%%%%%%%%%%%%%%%%%%%%%%%%%%%%%%%%%%%%%%%%%%%%%%%%%%%%%%%%%%%%%
    \item \docentry[gitaccount]{BlakeSchultze} : parent directory for all pCT code/data on Kodiak and the Tardis compute nodes
    \begin{ThinEnum}[parsep=\currentparsep]
        \item \docentry[gitrepo]{LaTeX-Packages} : provides the package \docentry![file]{my-latex.sty} which is included in TeX documents to provide access to the definitions of new commands/macros/environments, load the external/3rd-party package dependencies, and configure the typesetting of LaTeX documents as well as providing the collection of LaTeX style (.sty) and other files included in this repository upon which these definitions/configurations are dependent.
        \item \docentry[gitrepo]{pCT\_Documentation} : contains an expanded set of pCT documentation files with additional resources not included in the \docentry[gitrepo]{pCT-collaboration\dirsep pCT-docs} repository, such as pCT publications and theses/dissertations.
        \item \docentry[gitrepo]{pCT\_Reconstruction} : the original repository in which Baylor's pCT reconstruction program was developed, which also contains the experimental development of an alternative program configuration with several automated routines, and is currently being merged into the release version of Baylor's reconstruction program as provided in \docentry[gitrepo]{BaylorICTHUS\dirsep pCT\_Reconstruction}.
        \item \docentry[gitrepo]{WED\_Analysis} : provides tool for determining the water-equivalent depth (WED) for a set of beam-aim point (BAP) coordinates based on reconstructed image RSP values, using the voxel walk algorithm developed as part of the pCT reconstruction program.  This algorithm steps from voxel edge to voxel edge along a trajectory to determine exact voxel intersection coordinates and prevent the missing of small voxel intersections which can occur when taking constant length steps along a path as was done in the original reconstruction program.
    \end{ThinEnum}
\end{ThinEnum}
\end{tcbparbox}
%\par\bigskip\bigskip\bigskip\bigskip\bigskip\smallskip
\end{tcbenvironment}
\endinput
%\clearpage 
\abbrevTestContent
%-------------------------------------------------------------------------------------------------------------------------------%
%-------------------------------------------------------------------------------------------------------------------------------%
%-------------------------------------------------------------------------------------------------------------------------------%
%%\clearpage
\Chapter{File Lists}
%\clearpage
\Section(Reconstruction File List){\texttt{reconstruction\_data} File List}
%%%%%%%%%%%%%%%%%%%%%%%%%%%%%%%%%%%%%%%%%%%%%%%%%%%%%%%%%%%%%%%%%%%%%%%%%%%%%%%%%%%%%%%%%%%%%%%%%%%%%%%%%%%%%%%%%%%%%%%%%%%%%%%%%%%%%%%%%%%%%%%%%
%%%%%%%%%%%%%%%%%%%%%%%%%%%%%%%%%%%%%%%%%%%%%%%%%%%%%%%%%%%%%%%%%%%%%%%%%%%%%%%%%%%%%%%%%%%%%%%%%%%%%%%%%%%%%%%%%%%%%%%%%%%%%%%%%%%%%%%%%%%%%%%%%
%%%%%%%%%%%%%%%%%%%%%%%%%%%%%%%%%%%%%%%%%%%%%%%%%%%%%%%%%%%%%%%%%%%%%%%%%%%%%%%%%%%%%%%%%%%%%%%%%%%%%%%%%%%%%%%%%%%%%%%%%%%%%%%%%%%%%%%%%%%%%%%%%
\begin{tcbenvironment}|Reconstruction File List|
%%%%%%%%%%%%%%%%%%%%%%%%%%%%%%%%%%%%%%%%%%%%%%%%%%%%%%%%%%%%%%%%%%%%%%%%%%%%%%%%%%%%%%%%%%%%%%%%%%%%%%%%%%%%%%%%%%%%%%%%%%%%%%%%%%%%%%%%%%%%%%%%%
\begin{tcbContentsBox}
\tcbsectionheaderfont Below is a list of optional reconstruction data/image files not written to disk by default and not listed in the \autohyperlink[type=Subsection]{Reconstruction Data Hierarchy} section.
\end{tcbContentsBox}
%%%%%%%%%%%%%%%%%%%%%%%%%%%%%%%%%%%%%%%%%%%%%%%%%%%%%%%%%%%%%%%%%%%%%%%%%%%%%%%%%%%%%%%%%%%%%%%%%%%%%%%%%%%%%%%%%%%%%%%%%%%%%%%%%%%%%%%%%%%%%%%%%
\begin{tcbparbox}{tcbenumeratedStyle}
\begin{fileList}
    \item \fileentry{bin\_counts.txt}{linearized bin \# for each proton history, where linearized \coloredtext*{MidnightBlue}{bin \# = t\_bin + angle\_bin * T\_BINS + v\_bin * T\_BINS * ANGULAR\_BINS}.}
    \item \fileentry{coefficient.bin}{file containing the tabulated scattering coefficient values for $\boldsymbol{\Sigma_1}$/$\boldsymbol{\Sigma_2}$ for $\boldsymbol{u_2-u_1}$/$\boldsymbol{u_1}$ values}
    \item \fileentry{cos\_table.bin}{file containing the tabulated values of cosine function}
    \item \docentry[file]{execution\_log.csv} : global execution log containing entries with scan/object information and the settings/parameters used in reconstruction for each reconstructions performed to date, with new row entries added each time the reconstruction program is executed.
    \item \docentry[file]{FBP.txt} : text image of filtered back projection (FBP) image.
    \item \docentry[file]{FBP.png} : conversion of \docentry![file]{FBP.txt} to PNG image.
    \item \docentry[file]{FBP\_med\_filtered.txt} : text image result of applying median filter to the filtered back projection (FBP) image.
        \item \docentry[file]{FBP\_med\_filtered.png} : conversion of \docentry![file]{FBP\_avg\_filtered.txt} to PNG image.
        \item \docentry[file]{FBP\_avg\_filtered.txt} : text image result of applying average filter to the filtered back projection (FBP) image.
        \item \docentry[file]{FBP\_avg\_filtered.png} : conversion of \docentry![file]{FBP\_avg\_filtered.txt} to PNG image.
    \item \fileentry{histories.bin}{binary file specifying entry/exit coordinates/angles, bin number, gantry angle, and hull entry x/y/z voxel \# for each history entering hull.}
    \item \docentry[file]{hull\_avg\_filtered.txt} : text image result of applying average filter to the hull image.
    \item \docentry[file]{hull\_avg\_filtered.png} : conversion of \docentry![file]{hull\_avg\_filtered.txt} to PNG image.
    \item \fileentry{mean\_rel\_ut\_angle.txt}{mean relative ut angle ($\boldsymbol{\angle ut_{out} -\angle ut_{in}}$) by linearized bin \#.}
    \item \fileentry{mean\_rel\_uv\_angle.txt}{mean relative uv angle ($\boldsymbol{\angle uv_{out} -\angle uv_{in}}$) by linearized bin \#.}
    \item \fileentry{mean\_WEPL.txt}{mean WEPL value by linearized bin \#.}
    \item \fileentry{MLP.bin}{binary file with MLP path data for each history entering hull.}
    \item \fileentry{MSC\_counts.txt}{text image indicating the \# of times each voxel was identified as lying outside the object using Modified Space/Silhouette Carving with the xy plane of each slice stacked on each other.}
    \item \fileentry{MSC\_hull.txt}{text image of object hull in 1s/0s obtained using Modified Space/Silhouette Carving with the xy plane of each slice stacked on each other.}
    \item \docentry[file]{MSC\_hull.png} : conversion of \docentry![file]{MSC\_hull.txt} to PNG image.
    \item \fileentry{poly\_1\_2.bin}{file containing the tabulated MLP polynomial values with coefficients $\boldsymbol{\{1,2,3,4,5,6\}}$}
    \item \fileentry{poly\_2\_3.bin}{file containing the tabulated MLP polynomial values with coefficients $\boldsymbol{\{2,3,4,5,6,7\}}$}
    \item \fileentry{poly\_3\_4.bin}{file containing the tabulated MLP polynomial values with coefficients $\boldsymbol{\{3,4,5,6,7,8\}}$}
    \item \fileentry{poly\_2\_6.bin}{file containing the tabulated MLP polynomial values with coefficients $\boldsymbol{\{2,6,12,20,30,42\}}$}
    \item \fileentry{poly\_3\_12.bin}{file containing the tabulated MLP polynomial values with coefficients $\boldsymbol{\{3,12,30,60,105,168\}}$}
    \item \fileentry{SC\_hull.txt}{text image of object hull in 1s/0s obtained using Space/Silhouette Carving with the xy plane of each slice stacked on each other.}
    \item \docentry[file]{SC\_hull.png} : conversion of \docentry![file]{SC\_hull.txt} to PNG image.
    \item \fileentry{sinogram.txt}{mean WEPL after statistical cuts with the $\boldsymbol{t_{bin}}$ and angular bin $\boldsymbol{\theta_{bin}}$ plane for each vertical bin $\boldsymbol{v_{bin}}$ stacked on each other.}
    \item \fileentry{sin\_table.bin}{file containing the tabulated values of sine function}
    \item \fileentry{SM\_counts.txt}{text image indicating the \# of times each voxel was identified as lying outside the object using Space/Silhouette Modeling with the xy plane of each slice stacked on each other.}
    \item \fileentry{SM\_hull.txt}{text image of object hull in 1s/0s obtained using Space/Silhouette Modeling with the xy plane of each slice stacked on each other.}
    \item \docentry[file]{SM\_hull.png} : conversion of \docentry![file]{SM\_hull.txt} to PNG image.
    \item \fileentry{stddev\_rel\_ut\_angle.txt}{standard deviation of the relative ut angle ($\boldsymbol{\angle ut_{out} -\angle ut_{in}}$) by linearized bin \#.}
    \item \fileentry{stddev\_rel\_uv\_angle.txt}{standard deviation of the relative uv angle ($\boldsymbol{\angle uv_{out} -\angle uv_{in}}$) by linearized bin \#.}
    \item \fileentry{stddev\_WEPL.txt}{standard deviation of the WEPL value by linearized bin \#.}
    \item \docentry[file]{WEPL.bin} : binary file specifying WEPL value for each history entering hull.
\end{fileList}
\end{tcbparbox}
\end{tcbenvironment}
\endinput %%%%%%%%%%%%%%%%%%%%%%%%%%%%%%%%%%%%%%%%%%%%%%%%%%%%%%%%%%%%%%%%%%%%%%%%%%%%%%%%%%%%%%%%%%%%%%%%%%%%%%%%%%%%%%%%%%%%%%%%%%%%%%%%%%%%%%%%%%%%%%%%%
%\newpage
%%%%%%%%%%%%%%%%%%%%%%%%%%%%%%%%%%%%%%%%%%%%%%%%%%%%%%%%%%%%%%%%%%%%%%%%%%%%%%%%%%%%%%%%%%%%%%%%%%%%%%%%%%%%%%%%%%%%%%%%%%%%%%%%%%%%%%%%%%%%%%%%%

%\clearpage%
\Section{Master File List}
%%%%%%%%%%%%%%%%%%%%%%%%%%%%%%%%%%%%%%%%%%%%%%%%%%%%%%%%%%%%%%%%%%%%%%%%%%%%%%%%%%%%%%%%%%%%%%%%%%%%%%%%%%%%%%%%%%%%%%%%%%%%%%%%%%%%%%%%%%%%%%%%%
%%%%%%%%%%%%%%%%%%%%%%%%%%%%%%%%%%%%%%%%%%%%%%%%%%%%%%%%%%%%%%%%%%%%%%%%%%%%%%%%%%%%%%%%%%%%%%%%%%%%%%%%%%%%%%%%%%%%%%%%%%%%%%%%%%%%%%%%%%%%%%%%%
%%%%%%%%%%%%%%%%%%%%%%%%%%%%%%%%%%%%%%%%%%%%%%%%%%%%%%%%%%%%%%%%%%%%%%%%%%%%%%%%%%%%%%%%%%%%%%%%%%%%%%%%%%%%%%%%%%%%%%%%%%%%%%%%%%%%%%%%%%%%%%%%%
\begin{tcbenvironment}|Master File List|
%%%%%%%%%%%%%%%%%%%%%%%%%%%%%%%%%%%%%%%%%%%%%%%%%%%%%%%%%%%%%%%%%%%%%%%%%%%%%%%%%%%%%%%%%%%%%%%%%%%%%%%%%%%%%%%%%%%%%%%%%%%%%%%%%%%%%%%%%%%%%%%%%
\begin{tcbContentsBox}
\tcbsectionheaderfont Below is a master list of files and their descriptions in alphabetical order.
\end{tcbContentsBox}
%%%%%%%%%%%%%%%%%%%%%%%%%%%%%%%%%%%%%%%%%%%%%%%%%%%%%%%%%%%%%%%%%%%%%%%%%%%%%%%%%%%%%%%%%%%%%%%%%%%%%%%%%%%%%%%%%%%%%%%%%%%%%%%%%%%%%%%%%%%%%%%%%
\begin{tcbparbox}{tcbenumeratedStyle}
\begin{enumerate}
    \item \docentry[fileset]{$<$Phantom$>$\_XXXX[\_AAA]\_xxx.dat} : raw experimental data for the object named \docentry![subcategory]{$<$Phantom$>$}, from run \# \docentry![subcategory]{XXXX[\_AAA]}, where \docentry![subcategory]{XXXX} is a 4 digit \# with leading zeros, \docentry![subcategory]{\_AAA} are optional ``\emph{subcategory tag(s)}'' indicating, e.g., a continuous scan (\docentry![subcategory]{\_Cont}), phantom position/section (inferior (\docentry![subcategory]{\_Inf}) or superior (\docentry![subcategory]{\_Sup}), top (\docentry![subcategory]{\_Top}) or bottom (\docentry![subcategory]{\_Bot}), etc.), and \docentry![subcategory]{xxx} is the gantry angle at which the data was acquired.
    \item \docentry[fileset]{$<$Phantom$>$\_XXXX[\_AAA]\_xxx.dat.root.reco.root.bin} : preprocessed experimental data with tracker coordinates, recovery of missing hits when possible, and calibrated WEPL measurements for the object named \docentry![subcategory]{$<$Phantom$>$}, from run \# \docentry![subcategory]{XXXX[\_AAA]}, where \docentry![subcategory]{XXXX} is a 4 digit \# with leading zeros, \docentry![subcategory]{\_AAA} are optional ``\emph{subcategory tag(s)}'' indicating, e.g., a continuous scan (\docentry![subcategory]{\_Cont}), phantom position/section (inferior (\docentry![subcategory]{\_Inf}) or superior (\docentry![subcategory]{\_Sup}), top (\docentry![subcategory]{\_Top}) or bottom (\docentry![subcategory]{\_Bot}), etc.), and \docentry![subcategory]{xxx} is the gantry angle at which the data was acquired.
    \item \fileentry{bin\_counts.txt}{linearized bin \# for each proton history, where linearized \coloredtext*{MidnightBlue}{bin \# = t\_bin + angle\_bin * T\_BINS + v\_bin * T\_BINS * ANGULAR\_BINS}.}
    \item \fileentry{coefficient.bin}{file containing the tabulated scattering coefficient values for $\boldsymbol{\Sigma_1}$/$\boldsymbol{\Sigma_2}$ for $\boldsymbol{u_2-u_1}$/$\boldsymbol{u_1}$ values}
    \item \fileentry{cos\_table.bin}{file containing the tabulated values of cosine function}
    \item \docentry[file]{execution\_log.csv} : global execution log containing entries with scan/object information and the settings/parameters used in reconstruction for each reconstructions performed to date, with new row entries added each time the reconstruction program is executed.
    \item \docentry[file]{execution\_log.txt} : execution times for various portions of preprocessing and/or reconstruction and total program execution time.
    \item \docentry[file]{FBP.txt} : text image of filtered back projection (FBP) image.
    \item \docentry[file]{FBP.png} : conversion of \docentry![file]{FBP.txt} to PNG image.
    \item \docentry[file]{FBP\_med\_filtered.txt} : text image result of applying median filter to the filtered back projection (FBP) image.
    \item \docentry[file]{FBP\_med\_filtered.png} : conversion of \docentry![file]{FBP\_avg\_filtered.txt} to PNG image.
    \item \docentry[file]{FBP\_avg\_filtered.txt} : text image result of applying average filter to the filtered back projection (FBP) image.
    \item \docentry[file]{FBP\_avg\_filtered.png} : conversion of \docentry![file]{FBP\_avg\_filtered.txt} to PNG image.
    \item \fileentry{histories.bin}{sbinary file specifying entry/exit coordinates/angles, bin number, gantry angle, and hull entry x/y/z voxel \# for each history entering hull.}
    \item \docentry[file]{hull.txt} : text image of selected object hull in 1s/0s.
    \item \docentry[file]{hull.png} : conversion of \docentry![file]{hull.txt} to PNG image.
    \item \docentry[file]{hull\_avg\_filtered.txt} : text image result of applying average filter to the hull image.
    \item \docentry[file]{hull\_avg\_filtered.png} : conversion of \docentry![file]{hull\_avg\_filtered.txt} to PNG image.
    \item \fileentry{mean\_rel\_ut\_angle.txt}{mean relative ut angle ($\boldsymbol{\angle ut_{out} -\angle ut_{in}}$) by linearized bin \#.}
    \item \fileentry{mean\_rel\_uv\_angle.txt}{mean relative uv angle ($\boldsymbol{\angle uv_{out} -\angle uv_{in}}$) by linearized bin \#.}
    \item \fileentry{mean\_WEPL.txt}{mean WEPL value by linearized bin \#.}
    \item \fileentry{MLP.bin}{binary file with MLP path data for each history entering hull.}
    \item \fileentry{MSC\_counts.txt}{text image indicating the \# of times each voxel was identified as lying outside the object using Modified Space/Silhouette Carving.}
    \item \fileentry{MSC\_hull.txt}{text image of object hull in 1s/0s obtained using Modified Space/Silhouette Carving.}
    \item \docentry[file]{MSC\_hull.png} : conversion of \docentry![file]{MSC\_hull.txt} to PNG image.
    \item \fileentry{poly\_1\_2.bin}{file containing the tabulated MLP polynomial values with coefficients $\boldsymbol{\{1,2,3,4,5,6\}}$}
    \item \fileentry{poly\_2\_3.bin}{file containing the tabulated MLP polynomial values with coefficients $\boldsymbol{\{2,3,4,5,6,7\}}$}
    \item \fileentry{poly\_3\_4.bin}{file containing the tabulated MLP polynomial values with coefficients $\boldsymbol{\{3,4,5,6,7,8\}}$}
    \item \fileentry{poly\_2\_6.bin}{file containing the tabulated MLP polynomial values with coefficients $\boldsymbol{\{2,6,12,20,30,42\}}$}
    \item \fileentry{poly\_3\_12.bin}{file containing the tabulated MLP polynomial values with coefficients $\boldsymbol{\{3,12,30,60,105,168\}}$}
    \item \docentry[fileset]{projection\_xxx.bin} : preprocessed data files containing tracker coordinates and WEPL data for gantry angle $\text{\docentry![subcategory]{xxx}} =\{\text{\docentry![subcategory]{001}}, \text{\docentry![subcategory]{002}}, \text{\docentry![subcategory]{003}}, \cdots\}$ used as input to image reconstruction.
    \item \docentry[fileset]{raw\_xxx.bin} : binary files containing trigger/tracker/energy detector data from event builder associated with gantry angle $\text{\docentry[subcategory]{xxx}} =\{\text{\docentry[subcategory]{001}}, \text{\docentry![subcategory]{002}}, \text{\docentry[subcategory]{003}}, \cdots\}$.
    \item \docentry[file]{readme.txt} : contains input raw data info, phantom name, and run date.
    \item \docentry[file]{settings.cfg} : specifies scan properties such as gantry angle interval, t/v detector size, reconstruction volume dimensions, etc and default settings and parameters to use in reconstructing this data set.
    \item \docentry[file]{settings\_log.cfg} : copy of \docentry![file]{settings.cfg} with any changes made to parameters/options applied at execution, if any.
    \item \fileentry{SC\_hull.txt}{text image of object hull in 1s/0s obtained using Space/Silhouette Carving.}
    \item \docentry[file]{SC\_hull.png} : conversion of \docentry![file]{SC\_hull.txt} to PNG image.
    \item \fileentry{sinogram.txt}{mean WEPL after statistical cuts with the $\boldsymbol{t_{bin}}$ and angular bin $\boldsymbol{\theta_{bin}}$ plane for each vertical bin $\boldsymbol{v_{bin}}$ stacked on each other.}
    \item \fileentry{sin\_table.bin}{file containing the tabulated values of sine function}
    \item \fileentry{SM\_counts.txt}{text image indicating the \# of times each voxel was identified as lying outside the object using Space/Silhouette Modeling.}
    \item \fileentry{SM\_hull.txt}{text image of object hull in 1s/0s obtained using Space/Silhouette Modeling.}
    \item \docentry[file]{SM\_hull.png} : conversion of \docentry![file]{SM\_hull.txt} to PNG image.
    \item \fileentry{stddev\_rel\_ut\_angle.txt}{standard deviation of the relative ut angle ($\boldsymbol{\angle ut_{out} -\angle ut_{in}}$) by linearized bin \#.}
    \item \fileentry{stddev\_rel\_uv\_angle.txt}{standard deviation of the relative uv angle ($\boldsymbol{\angle uv_{out} -\angle uv_{in}}$) by linearized bin \#.}
    \item \fileentry{stddev\_WEPL.txt}{standard deviation of the WEPL value by linearized bin \#.}
    \item \docentry[file]{TVcorr.txt} : contains TV corrected WEPL calibration curve coefficients.
    \item \docentry[file]{TV\_measurements.txt} : contains total variation (TV) measurements before/after each iteration
    \item \docentry[file]{x\_0.txt} : text image of initial iterate.
    \item \docentry[file]{x\_0.png} : conversion of \docentry![file]{x\_0.txt} to PNG image.
%   \item \fileentry[documentation-fileset]{x\_k.dcm}{DICOM images of x after $k$ iterations.}
    \item \docentry[fileset]{x\_k.txt} : text image of reconstructed image $\boldsymbol{x^k}$ after $\boldsymbol{k}$ iterations.
    \item \docentry[fileset]{x\_k.png} : PNG image of reconstructed image $\boldsymbol{x^k}$ after $\boldsymbol{k}$ iterations.
    \item \docentry[file]{Wcalib.txt} : contains final WEPL calibration curve coefficients.
    \item \docentry[file]{WcalibTemp.txt} : temporary file containing WEPL calibration curve coefficients.
    \item \fileentry{WEPL.bin}{binary file specifying WEPL value for each history entering hull.}
\end{enumerate}
\end{tcbparbox}
\end{tcbenvironment}
\endinput %%%%%%%%%%%%%%%%%%%%%%%%%%%%%%%%%%%%%%%%%%%%%%%%%%%%%%%%%%%%%%%%%%%%%%%%%%%%%%%%%%%%%%%%%%%%%%%%%%%%%%%%%%%%%%%%%%%%%%%%%%%%%%%%%%%%%%%%%%%%%%%%%
%\newpage
%%%%%%%%%%%%%%%%%%%%%%%%%%%%%%%%%%%%%%%%%%%%%%%%%%%%%%%%%%%%%%%%%%%%%%%%%%%%%%%%%%%%%%%%%%%%%%%%%%%%%%%%%%%%%%%%%%%%%%%%%%%%%%%%%%%%%%%%%%%%%%%%%
%\clearpage\setcounter{page}{0}\thispagestyle{fancy}\pagenumbering{roman}
%-------------------------------------------------------------------------------------------------------------------------------%
%-------------------------------------------------------------------------------------------------------------------------------%
%-------------------------------------------------------------------------------------------------------------------------------%
\endinput