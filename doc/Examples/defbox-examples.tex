%-----------------------------------------------------------------------------------------------------------------------------------------%
\Section{Inline Text Boxes}
%-----------------------------------------------------------------------------------------------------------------------------------------%
\tcbinlinetextbox{\tcbinlinetextbox{\RaisedText{Nomenclature Entry}}}
\tcbinlinetextbox[inlineboxblue]{\BeveledText{Nomenclature Entry}}
\tcbinlinetextbox{Nomenclature Entry}
\tcbinlinetextbox[inlineboxltblue]{Nomenclature Entry}
\tcbinlinetextbox[inlineboxblue]{Nomenclature Entry}
%-------------------------------------------------------------------------------------------------------------------------------%
\Section{Itemized Text Boxes}
%-------------------------------------------------------------------------------------------------------------------------------%
\begin{tcbitembox}[width=0.45\columnwidth]{{personal}}
   %These are the menu items:
   %\begin{itemize}
   \tcblower
   \item One
   \item Two
   \item Three
   \item Four 
   \item Five
   %\end{itemize}
\end{tcbitembox}%%
%-------------------------------------------------------------------------------------------------------------------------------%
%-------------------------------------------------------------------------------------------------------------------------------%
%-------------------------------------------------------------------------------------------------------------------------------%
\Section{Definition Text Boxes}
%-----------------------------------------------------------------------------------------------------------------------------------------%
%\clearpage
\begin{MDFdefbox}{fullcornerbox}
    \blindtext%\lipsum[3]
\end{MDFdefbox}
%-------------------------------------------------------------------------------------------------------------------------------%
\begin{MDFdefbox}{cornerbox}
    \blindtext%\lipsum[3]
\end{MDFdefbox}
%-------------------------------------------------------------------------------------------------------------------------------%
\begin{MDFdefbox}{cornerbox}(Your Title)
    \blindtext%\lipsum[3]
\end{MDFdefbox}
%-------------------------------------------------------------------------------------------------------------------------------%
\begin{MDFdefbox}{borderbox}
    \blindtext%\lipsum[3]
\end{MDFdefbox}
%-------------------------------------------------------------------------------------------------------------------------------%
\begin{MDFdefbox}{titlebox}
    \blindtext%\lipsum[3]
\end{MDFdefbox}
%-------------------------------------------------------------------------------------------------------------------------------%
\begin{MDFdefbox}{tabbox}{Title Here}
    \blindtext%\lipsum[3]
\end{MDFdefbox}
%-------------------------------------------------------------------------------------------------------------------------------%
\begin{MDFdefbox}{tabbox}[sidebyside]{Title Here}
    \blindtext%\lipsum[2]
    \tcblower
    \blindtext%\lipsum[4]
\end{MDFdefbox}
%-------------------------------------------------------------------------------------------------------------------------------%
\begin{MDFdefbox}{tabbox}(Your Title)[sidebyside]{Title Here}
    \blindtext%\lipsum[2]
    \tcblower
    \blindtext%\lipsum[4]
\end{MDFdefbox}
%-------------------------------------------------------------------------------------------------------------------------------%
\begin{MDFdefbox}{tcborder}%(Your Title)[sidebyside]{Title Here}
    \blindtext%\lipsum[2]
\end{MDFdefbox}%%
%-----------------------------------------------------------------------------------------------------------------------------------------%
\begin{tcbnote}
    \blindtext%\lipsum[2]
\end{tcbnote}%%
%-------------------------------------------------------------------------------------------------------------------------------%
%-------------------------------------------------------------------------------------------------------------------------------%
%-------------------------------------------------------------------------------------------------------------------------------%
\Section{Definition/Theorem/Corollary/Lemma Title Boxes}
%-------------------------------------------------------------------------------------------------------------------------------%
\begin{tcbtheorem}{Mittelwertsatz $n$ Variable}{mittelwertsatz}
    Es sei $n\in\mathbb{N}$, $D\subseteq\mathbb{R}^n$ eine offene Menge und $f\in C^{1}(D,\mathbb{R})$. Dann gibt es auf jeder Strecke $[x_0,x]\subset D$ einen Punkt $\xi\in[x_0,x]$, so dass gilt
    \begin{equation*}
        f(x)-f(x_0) = \operatorname{grad} f(\xi)^{\top}(x-x_0)
    \end{equation*}
\end{tcbtheorem}
%-------------------------------------------------------------------------------------------------------------------------------%
\begin{tcbcorollary}{Mittelwertsatz $n$ Variable}{mittelwertsatz}
    Es sei $n\in\mathbb{N}$, $D\subseteq\mathbb{R}^n$ eine offene Menge und $f\in C^{1}(D,\mathbb{R})$. Dann gibt es auf jeder Strecke $[x_0,x]\subset D$ einen Punkt $\xi\in[x_0,x]$, so dass gilt
    \begin{equation*}
        f(x)-f(x_0) = \operatorname{grad} f(\xi)^{\top}(x-x_0)
    \end{equation*}
\end{tcbcorollary}
%-------------------------------------------------------------------------------------------------------------------------------%
\begin{tcbdefinition}{Mittelwertsatz $n$ Variable}{mittelwertsatz}
    Es sei $n\in\mathbb{N}$, $D\subseteq\mathbb{R}^n$ eine offene Menge und $f\in C^{1}(D,\mathbb{R})$. Dann gibt es auf jeder Strecke $[x_0,x]\subset D$ einen Punkt $\xi\in[x_0,x]$, so dass gilt
    \begin{equation*}
        f(x)-f(x_0) = \operatorname{grad} f(\xi)^{\top}(x-x_0)
    \end{equation*}
\end{tcbdefinition}
%-------------------------------------------------------------------------------------------------------------------------------%
\begin{tcblemma}{Mittelwertsatz $n$ Variable}{mittelwertsatz}
    Es sei $n\in\mathbb{N}$, $D\subseteq\mathbb{R}^n$ eine offene Menge und $f\in C^{1}(D,\mathbb{R})$. Dann gibt es auf jeder Strecke $[x_0,x]\subset D$ einen Punkt $\xi\in[x_0,x]$, so dass gilt
    \begin{equation*}
        f(x)-f(x_0) = \operatorname{grad} f(\xi)^{\top}(x-x_0)
    \end{equation*}
\end{tcblemma}
%-------------------------------------------------------------------------------------------------------------------------------%
%-------------------------------------------------------------------------------------------------------------------------------%
%-------------------------------------------------------------------------------------------------------------------------------%
\Section{Definition/Theorem/Corollary/Lemma Boxes}
%-------------------------------------------------------------------------------------------------------------------------------%
\begin{Theorem}{Mittelwertsatz $n$ Variable}{mittelwertsatz}
    Es sei $n\in\mathbb{N}$, $D\subseteq\mathbb{R}^n$ eine offene Menge und $f\in C^{1}(D,\mathbb{R})$. Dann gibt es auf jeder Strecke $[x_0,x]\subset D$ einen Punkt $\xi\in[x_0,x]$, so dass gilt
    \begin{equation*}
        f(x)-f(x_0) = \operatorname{grad} f(\xi)^{\top}(x-x_0)
    \end{equation*}
\end{Theorem}
%-------------------------------------------------------------------------------------------------------------------------------%
\begin{Corollary}{Nullstellenexistenz}{nullstellen}
Ist $f[a,b]\to\mathbb{R}$ stetig und haben $f(a)$ und $f(b)$ entgegengesetzte
Vorzeichen, also $f(a)f(b)<0$, so besitzt $f$ eine Nullstelle $x_0\in]a,b[$,
also $f(x_0)=0$.
    \begin{equation*}
        f(x)-f(x_0) = \operatorname{grad} f(\xi)^{\top}(x-x_0)
    \end{equation*}
\end{Corollary}
%-----------------------------------------------------------------------------------------------------------------------------------------%
\begin{Definition}{Differenzierbarkeit}{diffbarkeit}
    Eine Funktion $f~I\to\mathbb{R}$ auf einem Intervall $I$ hei\ss{}t in $x_0\in I$ differenzierbar oder linear approximierbar, wenn der Grenzwert
    \begin{equation*}
        \lim\limits_{x\to x_0}\frac{f(x)-f(x_0)}{x-x_0}=
        \lim\limits_{h\to 0}\frac{f(x_0+h)-f(x_0)}{h}
    \end{equation*}
    existiert. Bei Existenz hei\ss{}t dieser Grenzwert Ableitung oder Differential quotient von $f$ in $x_0$ und man schreibt f\"{u}r ihn
    \begin{equation*}
        f'(x_0)\quad\text{oder}\quad\frac{df}{dx}(x_0).
    \end{equation*}
\end{Definition}
%-------------------------------------------------------------------------------------------------------------------------------%
\begin{Lemma}{Mittelwertsatz $n$ Variable}{mittelwe}
    Es sei $n\in\mathbb{N}$, $D\subseteq\mathbb{R}^n$ eine offene Menge und $f\in C^{1}(D,\mathbb{R})$. Dann gibt es auf jeder Strecke $[x_0,x]\subset D$ einen Punkt $\xi\in[x_0,x]$, so dass gilt
    \begin{equation*}
        f(x)-f(x_0) = \operatorname{grad} f(\xi)^{\top}(x-x_0)
    \end{equation*}
\end{Lemma}
%-------------------------------------------------------------------------------------------------------------------------------%
%-------------------------------------------------------------------------------------------------------------------------------%
%-------------------------------------------------------------------------------------------------------------------------------%
%-------------------------------------------------------------------------------------------------------------------------------%
%\begin{theorem}[ Pythagorean theorem]
%
%\end{theorem}
%-------------------------------------------------------------------------------------------------------------------------------%
%-------------------------------------------------------------------------------------------------------------------------------%
%-------------------------------------------------------------------------------------------------------------------------------%
%\begin{definitioniBB}
%\lipsum[3]
%\end{definitioniBB}
%\begin{definitioniii}
%\lipsum[4]
%\end{definitioniii}
%\begin{definitionBB}
%\lipsum[4]
%\end{definitionBB}
%%\chapter(hello}
%\begin{ejemplo}{Title}
%  \lipsum[2]
%\end{ejemplo}
%\begin{ejemplo}[sidebyside]{Title}
%  \lipsum[2]
%  \tcblower
%  \lipsum[4]
%\end{ejemplo}
\endinput 