%\mathon%
    %\clearpage
%\formattheequation{equation}
%\formattheequation{problems}
%\formattheequation{sections}
%\makeatletter\the@prob@counter\makeatother
\begin{problem}[3](8)%
    For each of the following, determine whether the random process is (1) WSS or (2) m.s. ergodic in the mean.
\end{problem}
%\begin{tcbproof}
\begin{tcbsection}[type=proof]
    Beginning with the check of WSS
    \begin{align}
        \mu_x &= \E{x(n)} = \E{A(\zeta)} = \dfrac{1}{2}\quad \text{: constant}
        \\
        r_x(n_1,n_2) &= \E{x(n_1,\zeta)x(n_2,\zeta)}
        = \E{A^2(\zeta)}
        =\int\limits_0^1{x^2\,dx}
        =\dfrac{x^3}{3}\bigg\lvert_{x=0}^1
        =\dfrac{1}{3}\quad \text{: constant}\\
        r_x(n_1,n_2)
    &=
    \begin{dcases*}
        \E{x(n_1)x^*(n_2)} = \E{x(n_1)}\E{x(n_2)} = (2p-1)(2p-1) = (2p-1)^2             & if |n_1\ne n_2|\\
        \E{x(n)x^*(n)} = \sum x(n)x^*(n)\Pr*{X_n=x}  = (1)^2\cdot p + (-1)^2\cdot(1-p) = p + (1-p) = 1 &    if |n_1=n_2=n|
    \end{dcases*}
    \end{align}
    However, notice that the random variable $x(n,\zeta)=A(\zeta)$ is constant for a particular value of $\zeta$ and although its expected value is consistent with the mean of the process,  the sequence $x(n,\zeta)$ remains constant as $N\to\infty$ and therefore, the sample mean does not converge to the population mean $\mean{x}$.

\begin{tcbsolutionverb}*
%\mathversion{doublestruck}
Therefore, since $\E{\timeavg{x(n)}}$ does not converge to the population mean $\mean{x}$ and, hence, the process is not M.S.
ergodic in the mean.
\end{tcbsolutionverb}

Continuing with a check of whether the random variable is M.S. ergodic in the mean:
\begin{align}
     \E{\timeavg{x(n)}}
    &=
    \E
    {
        \dfrac{1}{2N+1}
        \sum\limits_{-N}^{N} x(n)
    }
    =
    \dfrac{1}{2N+1}
    \sum\limits_{-N}^{N}  \E{ x(n)}
    =
    \dfrac{1}{2N+1}
    \sum\limits_{-N}^{N}  \E{A(\zeta)}
        =
    \dfrac{1}{2}
    \\
    \E{\timeavg{x(n)}}
    &=\mu_x\quad\text{\ding{52}}
\end{align}
    \begin{tcbeqnlist}[style=darkblue, title=WSS Results]
       \tcbeqnlistboxentry
           {\mean{x} }
           {= \dfrac{1}{2}\quad \textbf{: constant}}
            \\
            \tcbeqnlistboxentry
               {r_x(n_1,n_2) }
               {=\dfrac{1}{3}\quad \textbf{: constant}}
            \\
            \tcbeqnlistboxentry
               {\Longrightarrow\quad X(t) }
               {= A(\zeta) \textbf{ \ul{is} WSS}}
     \end{tcbeqnlist}
\end{tcbsection}
%-------------------------------------------------------------------------------------------------------------------------------%
\begin{problem}>[4](2)%
    For each of the following, determine whether the random process is (1) WSS or (2) m.s. ergodic in the mean.
\end{problem}
%\begin{tcbsection}[type=solution]
\begin{solution}
    Beginning with the check of WSS
    \begin{align}
        \mu_x &= \E{x(n)} = \E{A(\zeta)} = \dfrac{1}{2}\quad \text{: constant}
        \\
        r_x(n_1,n_2) &= \E{x(n_1,\zeta)x(n_2,\zeta)}
        = \E{A^2(\zeta)}
        =\int\limits_0^1{x^2\,dx}
        =\dfrac{x^3}{3}\bigg\lvert_{x=0}^1
        =\dfrac{1}{3}\quad \text{: constant}\\
        \tcbequation{\mu_x}{= \E{A^2(\zeta)}}%
        \tcbequation(*){\mu_x}{= \E{A^2(\zeta)}}%
        \tcbequation{\mu_x}{= \E{A^2(\zeta)}}%
        \tcbequation(+!){\mu_x}{=\E{A^2(\zeta)}}{2}(-)%<west,east,45>%
        %
        r_x(n_1,n_2)
        &=
        \begin{dcases*}
            \E{x(n_1)x^*(n_2)} = \E{x(n_1)}\E{x(n_2)} = (2p-1)(2p-1) = (2p-1)^2             & if |n_1\ne n_2|\\
            \E{x(n)x^*(n)} = \sum x(n)x^*(n)\Pr*{X_n=x}  = (1)^2\cdot p + (-1)^2\cdot(1-p) = p + (1-p) = 1 &    if |n_1=n_2=n|
        \end{dcases*}
    \end{align}
    However, notice that the random variable $x(n,\zeta)=A(\zeta)$ is constant for a particular value of $\zeta$ and although its expected value is consistent with the mean of the process,  the sequence $x(n,\zeta)$ remains constant as $N\to\infty$ and therefore, the sample mean does not converge to the population mean $\mean{x}$.

\begin{tcbsolutionverb}*
%\mathversion{doublestruck}
Therefore, since  |\E{\timeavg{x(n)}}| does not converge to the population mean |\mean{x}| and, hence, the process is not M.S.
ergodic in the mean.
\end{tcbsolutionverb}

Continuing with a check of whether the random variable is M.S. ergodic in the mean:

\begin{align}
     \E{\timeavg{x(n)}}
    &=
    \E
    {
        \dfrac{1}{2N+1}
        \sum\limits_{-N}^{N} x(n)
    }
    =
    \dfrac{1}{2N+1}
    \sum\limits_{-N}^{N}  \E{ x(n)}
    =
    \dfrac{1}{2N+1}
    \sum\limits_{-N}^{N}  \E{A(\zeta)}
        =
    \dfrac{1}{2}
    \\
    \E{\timeavg{x(n)}}
    &=\mu_x\quad\text{\ding{52}}
\end{align}
\begin{tcbeqnlist}[style=blue, beforeskip=true, goldtrim=false]
    \tcbeqnlistboxentry
        {\text{Mean} = \mu_x = 1 > 0}
        {\implies\text{$f_x(x)$ is shifted to the right of the origin}}
        \\
        \tcbeqnlistboxentry
        {\text{Variance} = \var{x(\zeta)} \triangleq\sigma_x^2 = 1 > 0}
        {\implies\text{$f_x(x)$ has an equal spread in values as a standard Gaussian}}
        \\
        \tcbeqnlistboxentry
        {\text{Skewness} \triangleq\tilde{\kappa}_x^{(3)} = 2 > 0}
        {\implies\text{$f_x(x)$ leans right.}}
        \\
        \tcbeqnlistboxentry
        {\text{Kurtosis} = \tilde{\kappa}_x^{(4)} = 6 > 0}
        {\implies\text{$f_x(x)$ has a much sharper peak than a standard Gaussian}}
\end{tcbeqnlist}
\begin{tcbeqnlist}[style=solution, beforeskip=true, goldtrim=false]
       \tcbeqnlistboxentry
           {\mean{x} }
           {= \dfrac{1}{2}\quad \textbf{: constant}}
       \\
       \tcbeqnlistboxentry
               {r_x(n_1,n_2) }
               {=\dfrac{1}{3}\quad \textbf{: constant}}
       \\
       \tcbeqnlistboxentry
               {\Longrightarrow\quad X(t) }
               {= A(\zeta) \textbf{ \ul{is} WSS}}
\end{tcbeqnlist}
\begin{tcbeqnlist}[style=solution, beforeskip=true, title=Solution]
       \tcbeqnlistboxentry
           {\mean{x} }
           {= \dfrac{1}{2}\quad \textbf{: constant}}
            \\
            \tcbeqnlistboxentry
               {r_x(n_1,n_2) }
               {=\dfrac{1}{3}\quad \textbf{: constant}}
            \\
            \tcbeqnlistboxentry
               {\Longrightarrow\quad X(t) }
               {= A(\zeta) \textbf{ \ul{is} WSS}}
       \end{tcbeqnlist}
%\end{solution}
\end{solution}
%\end{tcbsection}
%-------------------------------------------------------------------------------------------------------------------------------%
%-------------------------------------------------------------------------------------------------------------------------------%
%-------------------------------------------------------------------------------------------------------------------------------%
\begin{problem}>[0](5)
    For each of the following, determine whether the random process is (1) WSS or (2) m.s. ergodic in the mean.
\end{problem}
\begin{problem}>
    For each of the following, determine whether the random process is (1) WSS or (2) m.s. ergodic in the mean.
\end{problem}
\begin{problem}.
    For each of the following, determine whether the random process is (1) WSS or (2) m.s. ergodic in the mean.
\end{problem}
\begin{problem}'+
    For each of the following, determine whether the random process is (1) WSS or (2) m.s. ergodic in the mean.
\end{problem}
\begin{problem}[3](12)
    For each of the following, determine whether the random process is (1) WSS or (2) m.s. ergodic in the mean.
\end{problem}
\begin{problem}.
    For each of the following, determine whether the random process is (1) WSS or (2) m.s. ergodic in the mean.
\end{problem}
\begin{problem}(10)
    For each of the following, determine whether the random process is (1) WSS or (2) m.s. ergodic in the mean.
\end{problem}
\begin{problem}'
    For each of the following, determine whether the random process is (1) WSS or (2) m.s. ergodic in the mean.
\end{problem}
\begin{problem}.+
    For each of the following, determine whether the random process is (1) WSS or (2) m.s. ergodic in the mean.
\end{problem}
\begin{problem}.+
    For each of the following, determine whether the random process is (1) WSS or (2) m.s. ergodic in the mean.
\end{problem}
%\mathoff%
%*******************************************************************************************************************************%
%*********************************************************** NOTES *************************************************************%
%*******************************************************************************************************************************%
%-------------------------------------------------------------------------------------------------------------------------------%
\endinput
%*******************************************************************************************************************************%
%*********************************************************** NOTES *************************************************************%
%*******************************************************************************************************************************%
%-------------------------------------------------------------------------------------------------------------------------------%
%[length = 12pt, fill=green, width=5pt]{Latex[reversed]}
        &=\tcbhighmath[remember,overlay={%{\@IBP@diag@arrow@tail}-{\@IBP@arrow@head}south
\draw[tcbequationArrows] (@tcbequation@boxii.east) to[bend right=-45] ([yshift=0mm]frame.east);}]
{1-\frac{1}{x}.}\\ 