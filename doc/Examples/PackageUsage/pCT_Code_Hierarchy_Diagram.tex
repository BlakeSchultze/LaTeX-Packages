%-------------------------------------------------------------------------------------------------------------------------------%
\Section(pCT code Hierarchy Diagram){\texttt{pCT\_code} Hierarchy Diagram}%
[%
	the diagram below shows the hierarchy of \docentry[constdir]{pCT\_code} directories listed and described in the \autohyperlink[type=Subsection]{pCT Code Hierarchy} section, including the clones of the most commonly used GitHub accounts/repositories relevant to pCT (described in the \autohyperlink[type=Section]{GitHub Accounts/Repositories} section).%
]%
%-------------------------------------------------------------------------------------------------------------------------------%
%-------------------------------------------------------------------------------------------------------------------------------%
%-------------------------------------------------------------------------------------------------------------------------------%
%\centering%
\begin{tcbdiagram}
%\centering%
\matrix[matrix of nodes, row sep=4mm,column sep=1.5mm]
{%
&\node(parentdot)[pt]{};\\
&\node(01)[D]{ion};\\
&\node(11)[D]{pCT\_code};\\
\node(20)[pt]{};&\node(21)[pt]{};&&\node(23)[pt]{};\\
\node(30)[D]{user\_code};    &&&\node(33)[D]{git};\\[0mm]
\node(40)[D]{$<$username$>$};&\node(41)[pt]{};&\node(42)[pt]{};&\node(43)[pt]{};&&\node(45)[pt]{};\\
&\node(51)[D]{BlakeSchultze};&\node(52)[D]{BaylorICTHUS};&&&\node(55)[D]{pCT-collaboration};\\[1mm]
\node(60)[pt]{};&\node(61)[pt]{};&\node(62)[F]{pCT\_Reconstruction};&\node(63)[pt]{};&\node(64)[pt]{};&\node(65)[pt]{};&\node(66)[pt]{};&\node(67)[pt]{};\\[1mm]
\node(70)[F]{pCT\_Reconstruction};&\node(71)[F]{pCT\_Documentation};&&\node(73)[F]{pCT-docs};&\node(74)[F]{pCT\_Tools};&\node(75)[F]{Reconstruction\_BU};&\node(76)[F]{pct-recon-copy};&\node(77)[F]{Preprocessing};\\[1mm]
};
\graph[use existing nodes, edge label, edges=thick]
{
(parentdot)->(01)->(11)->(21)--(20)->(30)->(40);
(21)--(23)->(33)--(43)--(42)--(41)->(51)--(61)--(60)->(70);
(43)--(45)->(55)--(65)--(63)--(67)->(77);
(42)->(52)->(62);
(61)->(71);
(63)->(73);
(64)->(74);
(65)->(75);
(66)->(76);
};
\end{tcbdiagram}
%*******************************************************************************************************************************%
%*******************************************************************************************************************************%
%*******************************************************************************************************************************%
\endinput 