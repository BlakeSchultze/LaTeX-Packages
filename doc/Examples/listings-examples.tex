%-------------------------------------------------------------------------------------------------------------------------------%
%-------------------------------------------------------------------------------------------------------------------------------%
\Section{Code Listing Examples}
%-------------------------------------------------------------------------------------------------------------------------------%
%-------------------------------------------------------------------------------------------------------------------------------%
%-------------------------------------------------------------------------------------------------------------------------------%
\Subsection{Inline Terminal/Command Box Examples}
%-------------------------------------------------------------------------------------------------------------------------------%
\tcbinlinebashbox{cd Documents}\\
\tcbinlinebashbox*{cd Documents}\\
\tcbinlinebashbox!{git pull --rebase}\\
%-------------------------------------------------------------------------------------------------------------------------------%
\commandbox>{cd "My Documents"} changes to directory \commandbox{My Documents}.\\
\commandbox*>{dir /A} lists the directory content.\\
\commandbox*+>{copy example.txt d:\target} copies \commandbox*+{example.txt} to \commandbox+{d:\target}.
%-------------------------------------------------------------------------------------------------------------------------------%
\Subsection{Matlab/Terminal Box Examples}
%-------------------------------------------------------------------------------------------------------------------------------%
%\Matlab[hello]{1}{1}{22}{Matlab implementation of Newton-type algorithm for $(a_1,a_2)=(-1.5955, 0.95)$ and $\mu=0.1$.}{lst:103ab}{C:/Users/Blake/Documents/LaTeX/Testing/Test_Code/NewtonAlg.m}
%-------------------------------------------------------------------------------------------------------------------------------%
\Matlab[hello]{1}{1}{22}{Matlab implementation of Newton-type algorithm for $(a_1,a_2)=(-1.5955, 0.95)$ and $\mu=0.1$.}{lst:103ab}{C:/Users/Blake/Documents/LaTeX/Testing/Test_Code/NewtonAlg.m}
%-------------------------------------------------------------------------------------------------------------------------------%
\begin{tcbbashbox}![hello]%
(
    The option '-a' automatically stages all tracked/modified files before the commit.
    This can be combined with the message option '-m'.
)
    cd figure
    mkdir commit -am
\end{tcbbashbox}
%-------------------------------------------------------------------------------------------------------------------------------%
%-------------------------------------------------------------------------------------------------------------------------------%
\Matlab[hello]{1}{1}{22}{Matlab implementation of Newton-type algorithm for $(a_1,a_2)=(-1.5955, 0.95)$ and $\mu=0.1$.}{lst:104ab}{C:/Users/Blake/Documents/LaTeX/Testing/Test_Code/NewtonAlg.m}
%-------------------------------------------------------------------------------------------------------------------------------%
%\clearpage
\Matlab{1}{1}{22}{Matlab implementation of Newton-type algorithm for $(a_1,a_2)=(-1.5955, 0.95)$ and $\mu=0.1$.}{lst:105ab}{C:/Users/Blake/Documents/LaTeX/Testing/Test_Code/NewtonAlg.m}
%-------------------------------------------------------------------------------------------------------------------------------%
%*******************************************************************************************************************************%
%*******************************************************************************************************************************%
%*********************************************************************************************************************************************************%
%\clearpage
\begin{tcbbashbox}*%
(
    The option '-a' automatically stages all tracked and modified files before the commit.
    This can be combined with the message option '-m' as seen in the third line.
)
    cd figure
    dir commit
    rmdir myfolder
    cp demofile demofile.bak
    rm commit -a
    mv rmfbf sffsknb
    mkdir commit -am
\end{tcbbashbox}
%-------------------------------------------------------------------------------------------------------------------------------%
%\clearpage
\begin{tcbbashbox}[hello]%
(
    The option '-a' automatically stages all tracked and modified files before the commit.
    This can be combined with the message option '-m' as seen in the third line.
)
    cd figure
    dir commit
    rmdir myfolder
    cp demofile demofile.bak
    rm commit -a
    mv rmfbf sffsknb
    mkdir commit -am
\end{tcbbashbox}
%-------------------------------------------------------------------------------------------------------------------------------%
%\clearpage
\begin{tcbbashbox}*!%
(
    The option '-a' automatically stages all tracked/modified files before the commit.
    This can be combined with the message option '-m'.
)
    mkdir commit -am%
\end{tcbbashbox}
%-------------------------------------------------------------------------------------------------------------------------------%
%\clearpage
\begin{tcbbashbox}!
    mkdir commit -am%
\end{tcbbashbox}
%-------------------------------------------------------------------------------------------------------------------------------%
%*******************************************************************************************************************************%
%*******************************************************************************************************************************%
%*********************************************************************************************************************************************************%
%-------------------------------------------------------------------------------------------------------------------------------%
%-------------------------------------------------------------------------------------------------------------------------------%
%\tcbinlinebashbox{cd Documents}
%\tcbinlinebashbox!{git pull --rebase}
%\commandbox
%%-------------------------------------------------------------------------------------------------------------------------------%
%\begin{script}
%   >> help sin
%   sin Sine of argument in radians.
%       sin(X) is the sine of the elements of X.
%
%   See also asin, sind.
%
%   Overloaded methods:
%       sdpvar/sin
%       codistributed/sin
%       gpuArray/sin
%
%   Reference page in Help browser
%       doc sin
%   >>
%\end{script}
%-------------------------------------------------------------------------------------------------------------------------------%
%-------------------------------------------------------------------------------------------------------------------------------%
%-------------------------------------------------------------------------------------------------------------------------------%
\endinput
